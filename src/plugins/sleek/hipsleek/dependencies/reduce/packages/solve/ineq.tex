\documentstyle[12pt]{article}
\begin{document}
\begin{center} {\Large INEQ} \end{center}

\begin{center} Herbert Melenk \\ Konrad-Zuse-Zentrum fuer
Informationstechnik \\
Takustra\"se 7 \\ 
D14195 Berlin -- Dahlem\\ Germany \\ 
melenk@zib.de \end{center}

This package supports the operator {\bf ineq\_solve} that 
tries to solves single inequalities and sets of coupled inequalities.
The following types of systems are supported
\footnote{For linear optimization problems please use the operator
{\bf simplex} of the {\bf linalg} package}:
\begin{itemize}
\item only numeric coefficients (no parametric system),
\item a linear system of mixed equations and $<=$ -- $>=$ 
     inequalities, applying the method of Fourier and Motzkin
     \footnote{described by G.B. Dantzig in {\em Linear Programming 
      and Extensions.}},
\item a univariate inequality with $<=$, $>=$, $>$ or $<$ operator
     and polynomial or rational left--hand and right--hand sides,
     or a system of such inequalities with only one variable.
\end{itemize}

Syntax:
\begin{center}
{\tt INEQ\_SOLVE($<$expr$>$ [,$<$vl$>$])}
\end{center}
where $<$expr$>$ is an inequality or a list of coupled inequalities
and equations, and the optional argument $<$vl$>$ is a single
variable (kernel) or a list of variables (kernels). If not
specified, they are extracted automatically from $<$expr$>$.
For multivariate input an explicit variable list specifies the
elimination sequence: the last member is the most specific one.

An error message occurs if the input cannot be processed by the
currently implemented algorithms.

The result is a list. It is empty if the system has no feasible solution.
Otherwise the result presents the admissible ranges as set
of equations where each variable is equated to 
one expression or to an interval. 
The most specific variable is the first one in the result list and
each form contains only preceding variables (resolved form).
The interval limits can be formal {\bf max} or {\bf min} expressions.
Algebraic numbers are encoded as rounded number approximations.

\noindent
{\bf Examples}:

\begin{verbatim}
ineq_solve({(2*x^2+x-1)/(x-1) >=  (x+1/2)^2, x>0});

{x=(0 .. 0.326583),x=(1 .. 2.56777)}

 reg:=
 {a + b - c>=0, a - b + c>=0, - a + b + c>=0, 0>=0, 2>=0,
  2*c - 2>=0, a - b + c>=0, a + b - c>=0, - a + b + c - 2>=0,
  2>=0, 0>=0, 2*b - 2>=0, k + 1>=0, - a - b - c + k>=0,
   - a - b - c + k + 2>=0, - 2*b + k>=0,
   - 2*c + k>=0, a + b + c - k>=0,
  2*b + 2*c - k - 2>=0, a + b + c - k>=0}$

ineq_solve (reg,{k,a,b,c});

{c=(1 .. infinity),

 b=(1 .. infinity),

 a=(max( - b + c,b - c) .. b + c - 2),

 k=a + b + c}
\end{verbatim}

\end{document}

