\chapter{Introduction}

This document describes the \textsf{Frama-C/WP} plug-in that uses external decision procedures
to prove \textsf{ACSL} annotations of \textsf{C} functions.

The \textsf{WP} plug-in is named after \emph{Weakest Precondition}
calculus, a technique used to prove program properties initiated by
Hoare~\cite{Hoare1969}, Floyd~\cite{Floyd1967} and
Dijkstra~\cite{Dijkstra1968}. Recent tools implement this technique
with great performance, for instance \textsf{Boogie}~\cite{Leino2008}
and \textsf{Why}~\cite{Filliatre2003}. There is already a
\textsf{Frama-C} plug-in, \textsf{Jessie}~\cite{Jessie2009}, developed
at \textsc{INRIA}, that implements a weakest precondition calculus for
\textsf{C} programs by compiling them into the \textsf{Why} language.

The \textsf{WP} plug-in is a novel implementation of such a
\emph{Weakest Precondition} calculus for annotated \textsf{C}
programs, which focuses on parametrization \emph{w.r.t} the memory model.
It is a complementary work to the \textsf{Jessie} plug-in, which relies on a
separation memory model in the spirit of Burstall's
work~\cite{Burstall1972}.  The \textsf{Jessie} memory model is very
efficient for a large variety of well structured \textsf{C}-programs.
However, it does not apply when low-level memory manipulations, such as
heterogeneous casts, are involved. Moreover, \textsf{Jessie}
operates by compiling the \textsf{C} program to \textsf{Why}, a
solution that prevents the user from combining \textit{weakest
  precondition calculus} with other techniques, such as the
\textsf{Eva} analysis plug-in.

The \textsf{WP} plug-in has been designed with cooperation in
mind. That is, you may use \textsf{WP} for proving some annotations of
your \textsf{C} programs, and prove others with
different plug-ins. The recent improvements of the \textsf{Frama-C} kernel
are then responsible for managing such partial proofs and consolidating
them altogether.

This manual is divided into three parts. This first chapter introduces
the \textsf{WP} plug-in, the \emph{Weakest Precondition} calculus and
\emph{Memory Models}. Then, Chapter~\ref{wp-plugin} details how to use
and tune the plug-in within the \textsf{Frama-C}
platform. Chapter~\ref{wp-models} provides a description for the
included memory models. We present in Chapter~\ref{wp-simplifier} the
simplifier module and the efficient weakest precondition engine implemented
in the \textsf{WP} plug-in. Finally, in Chapter~\ref{wp-builtins}, we
provide additional information on the supported ACSL built-in symbols.

\clearpage

%-----------------------------------------------------------------------------
\section{Installation}
\label{wp-install}
%-----------------------------------------------------------------------------

The \textsf{WP} plug-in is distributed with the \textsf{Frama-C}
platform. However, it also requires the \textsf{Why-3} platform and
you should install at least one external prover in
order to fulfill proof obligations. An easy choice is to install the
\textsf{Alt-Ergo} theorem prover originally developed at
\textsc{inria} and now by \textsc{OcamlPro}\footnote{\textsf{Alt-Ergo}:
  \url{https://alt-ergo.ocamlpro.com/}}. When using the \textsf{Opam} package
  manager, these tools are automatically installed with \textsf{Frama-C}.

When updating or installing \textsf{Why-3}, the following command should be
run to detect available provers:

\begin{lstlisting}[basicstyle=\ttfamily]
rm -f ~/.why3.conf ; why3 config detect
\end{lstlisting}

See the documentation of \textsf{Why-3} to install other provers.


%-----------------------------------------------------------------------------
\section{Tutorial}
\label{wp-tutorial}
%-----------------------------------------------------------------------------

Consider the very simple example of a function that swaps the values
of two integers passed by reference:

\listingname{swap.c}
\cinput{../../tests/wp_manual/working_dir/swap.c}

A simple, although incomplete, \textsf{ACSL} contract for this function can be:

\listingname{swap1.h}
\cinput{../../tests/wp_manual/working_dir/swap1.h}

You can run \textsf{wp} on this example with:
\logsinput{swap1.log-file}

As expected, running \textsf{WP} for the \texttt{swap} contract
results in two \emph{proof obligations} (one for each '\verb+ensures+'
clause). The first one is discharged internally by the \textsf{Qed}
simplifier of \textsf{WP}, the second one is terminated by
\textsf{Alt-Ergo}.

You should notice the warning ``Missing RTE guards'', emitted by the
\textsf{WP} plug-in.  That is, the \emph{weakest precondition
  calculus} implemented in \textsf{WP} relies on the hypothesis that
your program is runtime-error free. In this example, the \texttt{swap}
function dereferences its two parameters, and these two pointers
should be valid.

By default, the \textsf{WP} plug-in does not generate any proof obligation
for verifying the absence of runtime errors in your code. Absence of runtime errors
can be proved with other techniques, for instance by running
the \textsf{Eva} plug-in, or by generating all the necessary annotations
with the \textsf{RTE} plug-in.

The simple contract for the \texttt{swap} function above is not strong enough to
prevent runtime errors: pointers given as parameters might be invalid.
Consider now the following new contract for \texttt{swap}:

\listingname{swap2.h}
\cinput{../../tests/wp_manual/working_dir/swap2.h}

For simplicity, the \textsf{WP} plug-in is able to run the
\textsf{RTE} plug-in for you, through the \texttt{-wp-rte} option.
Now, \textsf{WP} reports that the function \textsf{swap} completely fulfills its contract,
without any warnings regarding potential residual runtime errors:
\logsinput{swap2.log-file}

\clearpage
We have finished the job of validating this simple \textsf{C} program with
respect to its specification, as reported by the \emph{report} plug-in
that displays a consolidation status of all annotations:
\logsinput{swap3.log-file}
\clearpage

%-----------------------------------------------------------------------------
\section{Weakest Preconditions}
\label{wp-intro-calculus}
%-----------------------------------------------------------------------------

The principles of \emph{weakest precondition calculus} are quite
simple in essence. Given a code annotation of your program, say, an
assertion $Q$ after a statement $\mathit{stmt}$, the weakest precondition of $P$
is by definition the ``simplest'' property $P$ that must be valid
before $\mathit{stmt}$ such that $Q$ holds after the execution of $\mathit{stmt}$.

\paragraph{Hoare triples.}
In mathematical terms, we denote such a property by a Hoare triple:
$$ \{P\}\,\mathit{stmt}\,\{Q\} $$
which reads: \emph{``whenever $P$ holds, then after running $\mathit{stmt}$, $Q$ holds''}.

Thus, we can define the weakest precondition as a function $\mathit{wp}$ over
statements and properties such that the following Hoare triple always holds:

$$ \{\mathit{wp}(stmt,Q)\}\quad \mathit{stmt} \quad \{Q\} $$

For instance, consider a simple assignment over an integer local
variable $x$; we have:

$$ \{ x+1 > 0 \} \quad \mathtt{x = x+1;} \quad \{ x > 0 \} $$

It should be intuitive that in this simple case, the \emph{weakest
  precondition} for this assignment of a property $Q$ over $x$ can be
obtained by replacing $x$ with $x+1$ in $Q$.  More generally, for any
statement and any property, it is possible to define such a weakest
precondition.

\paragraph{Verification.}
Consider now function contracts. We basically have
\emph{pre-conditions}, \emph{assertions} and
\emph{post-conditions}. Say function $f$ has a precondition $P$ and a
post condition $Q$, we now want to prove that $f$ satisfies its
contract, which can be formalized by:

$$ \{P\}\quad f \quad \{Q\} $$

Introducing $W = \mathit{wp}(f,Q)$, we have by definition of $\mathit{wp}$:

$$ \{W\}\quad f \quad \{Q\} $$

Suppose now that we can \emph{prove} that precondition $P$ entails
weakest precondition $W$; we can then conclude that $P$ precondition
of $f$ always entails its postcondition $Q$. This proof can be summarized by
the following diagram:

$$
\frac%
{\quad(P \Longrightarrow W) \quad\quad \{W\}\,f\,\{Q\} \quad}%
{\{P\}\,f\,\{Q\}}
$$

This is the main idea of how to prove a property by weakest
precondition computation. Consider an annotation $Q$, compute its weakest
precondition $W$ across all the statements from $Q$ up to the
beginning of the function. Then, submit the property $P \Longrightarrow
W$ to a theorem prover, where $P$ are the preconditions of the
function. If this proof obligation is discharged, then one may conclude
the annotation $Q$ is valid for all executions.

\paragraph{Termination.}
We must point out a detail about program termination. Strictly
speaking, the \emph{weakest precondition} of property $Q$ through
statement $\mathit{stmt}$ should also ensure termination and execution without
runtime errors.

When the \verb+terminates+ clause is specified in function contracts,
\textsf{WP} generates proof
obligations to make sure that all loops have loop variants and that all function
calls terminate whenever necessary (thus requiring \verb+decreases+ clause for
recursive functions). Furthermore, \texttt{exit} behaviors of a function are
correctly handled by \textsf{WP}. More details about proof of termination in
\textsf{WP} are provided in Section~\ref{ss-sec:plugin-terminates}.

Regarding runtime errors, the proof obligations generated by
\textsf{WP} assume your program never raises any of them. As
illustrated in the short tutorial example of
section~\ref{wp-tutorial}, you should enforce the absence of runtime
errors on your own, for instance by running the \textsf{Eva}
plug-in or the \textsf{RTE} one and proving the generated assertions.

%-----------------------------------------------------------------------------
\section{Memory Models}
\label{wp-intro-models}
%-----------------------------------------------------------------------------

The essence of a \emph{weakest precondition calculus} is to translate
code annotations into mathematical properties. Consider the simple case
of a property about an integer \textsf{C}-variable \texttt{x}:
\begin{ccode}
  x = x+1;
  //@ assert P: x >= 0 ;
\end{ccode}

We can translate $P$ into the mathematical property $P(X)=X \geq 0$,
where $X$ stands for the value of variable \texttt{x} at the
appropriate program point. In this simple case, the effect of
statement \texttt{x=x+1} over $P$ is actually the substitution
$X\mapsto X+1$, that is $X+1\geq 0$.

The problem when applying \emph{weakest precondition calculus} to
\textsf{C} programs is dealing with \emph{pointers}. Consider now:
\begin{ccode}
  p = &x ;
  x = x+1;
  //@ assert Q: *p >= 0 ;
\end{ccode}

It is clear that, taking into account the aliasing between \texttt{*p}
and \texttt{x}, the effect of the increment of \texttt{x} cannot be
translated by a simple substitution of $X$ in $Q$.

This is where \emph{memory models} come to rescue.

A memory model defines a mapping from values inside the \textsf{C} memory
heap to mathematical terms. The \textsf{WP} has been designed to support
different memory models. There are currently three memory models
implemented, and we plan to implement new ones in future releases.
Those three models are all different from the one in the \textsf{Jessie}
plug-in, which makes \textsf{WP} complementary to \textsf{Jessie}.

\begin{description}

\item[\texttt{Hoare} model.]  A very efficient model that generates
  concise proof obligations. It simply maps each \textsf{C} variable to one
  pure logical variable.\par

  However, the heap cannot be represented in this model, and
  expressions such as \texttt{*p} cannot be translated at all. You
  can still represent pointer values, but you cannot read or write
  the heap through pointers.

\item[\texttt{Typed} model.] The default model for \textsf{WP}
  plug-in. Heap values are stored in several \emph{separated} global
  arrays, one for each atomic type (integers, floats, pointers) and
  an additional one for memory allocation. Pointer values are translated
  into an index into these arrays.\par

  In order to generate reasonable proof obligations, the values stored
  in the global array are not the machine ones, but the logical
  ones. Hence, all \textsf{C} integer types are represented by
  mathematical integers and each pointer type to a given type is represented
  by a specific logical abstract datatype.\par

  A consequence of having separated arrays is that heterogeneous casts
  of pointers cannot be translated in this model. For instance within
  this memory model, you cannot cast a pointer to \texttt{int} into a
  pointer to \texttt{char}, and then access the internal
  representation of the original \texttt{int} value into memory.

  However, variants of the \texttt{Typed} model enable limited forms of casts.
  See chapter~\ref{wp-models} for details.

\item[\texttt{Bytes} model. (Not Implemented Yet).] This is a
  low-level memory model, where the heap is represented as a wide
  array of bytes. Pointer values are exactly translated into memory
  addresses. Read and write operations from/to the heap are translated
  into manipulation of ranges of bits in the heap.\par

  This model is very \emph{precise} in the sense that all the details
  of the program are represented. This comes at the cost of huge proof
  obligations that are very difficult to discharge by automated
  provers, and generally require an interactive proof assistant.

\end{description}

Thus, each \emph{memory model} offers a different trade-off between
expressive power and ease of discharging proof obligations. The
\texttt{Hoare} memory model is very restricted but generates easy proof
obligations, \texttt{Bytes} is very expressive but generates difficult
proof obligations, and \texttt{Typed} offers an intermediate solution.

Chapter~\ref{wp-models} is dedicated to a more detailed description of
memory models, and how the \textsf{WP} plug-in uses and \emph{combines}
them to generate efficient proof obligations.

% \paragraph{Remark.}
% The original \texttt{Store} and \texttt{Runtime} memory models are no
% longer available since \textsf{WP} version \verb+0.7+ (Fluorine); the \texttt{Typed} model
% replaces the \texttt{Store} one; the \texttt{Runtime} model will be entirely
% re-implemented as \texttt{Bytes} model in some future release.

\section{Arithmetic Models}
\label{wp-model-arith}

The \textsf{WP} plug-in is able to take into account the precise
semantics of integral and floating-point operations of \textsf{C}
programs. However, doing so generally leads to very complex proof obligations.

For tackling this complexity, the \textsf{WP} plug-in relies on several
\textsf{arithmetic} models:

\begin{description}

\item[Machine Integer Model:] The kernel options are used to determine if an
  operation is allowed to overflow or not. In case where overflows or downcasts
  are forbidden, the model uses mathematical operators in place of modulo ones
  to perform the computations.

  For example, with kernel \emph{default} settings, addition of regular
  \texttt{signed int} values is interpreted by mathematical addition over
  unbounded \texttt{integers} ; and the addition of two
  \texttt{unsigned int} is interpreted by the addition modulo $2^{32}$.

  The user shall set the kernel options \texttt{-(no)-warn-xxx} to precisely tune the
  model. Using \texttt{-rte} or \texttt{-wp-rte} will
  generate all the necessary assertions to be verified.

\item[Natural Model:] integer operations are performed on mathematical
  integers. In \textsf{ACSL}, explicit conversions between different integer
  types are still translated with \emph{modulo}, though. Size of integers is
  also removed from type constrained, which avoids provers to run into deep
  exploration of large integer domains. Only signed-ness is kept from type
  constraints.

  Except for the lesser constrained type assumptions, this model behaves like
  the machine integer one when all kernel options \texttt{-warn-xxx} are set.
  However, the model \emph{does not} modifies them. Hence, using \texttt{-rte}
  or \texttt{-wp-rte} will generate a warning if some annotation might be not
  generated.

\item[Float Model:] floating-point values are represent in a special
  theory with dedicated operations over \texttt{float} and \texttt{double}
  values and conversion from and to their \texttt{real} representation \emph{via}
  rounding, as defined by the \textsf{C/ACSL} semantics.

  Although correct with respect to the \textsc{IEEE} specifications, this
  model still provides very little support for proving properties with automated
  provers. You may add additional properties using \emph{drivers}
  as explained later.

\item[Real Model:] floating-point operations are \emph{transformed} on
  reals, with \emph{no} rounding. This is completely unsound with
  respect to \textsf{C} and \textsf{IEEE} semantics. There is no way
  of recovering a correct or partial property from the generated proof
  obligations on floating-point operations with this model.

\end{description}

\paragraph{Remark:} with all models, there are conditions to meet for WP
proofs to be correct. Depending on the model used and the kernel options, those conditions
may change. WP do not generate proof obligations for runtime errors on its own. Instead, it can
discharge the annotations generated by the \textsf{Eva} analysis plug-in, or by the \textsf{RTE} plug-in.
Consider also using \texttt{-wp-rte} option.

\section{Limitations \& Roadmap}

The ambition of \textsf{WP} plug-in is to cover as many \textsf{ACSL} features as possible. However, some of them are still not available yet, because of lack of manpower or more fundamental reasons. This section provides an overview of those limitations, with roadmap indications: \textit{easy}
means that support could be provided on demand, \textit{medium} means that
significant manpower is required to implement the feature and \textit{hard} means that a mid-term research project would be required. This list of limitations is probably not exhaustive (yet) and will be maintained over future versions of \textsf{Frama-C/WP}.

\begin{description}
\item[Global invariants.]
Not implemented yet (\textit{easy}).
\item[Type invariants.]
Type invariants requires to be coupled with new memory models and
some memory region analysis (\textit{hard}).
\item[Model fields.]
This \textsf{ACSL} feature is generally coupled with type invariants and global invariants, hence it is not implemented yet. From a practical point
of view, we think that ghost fields with logic types would be very complementary and easier to use. We are waiting for challenging use cases
to implement these features (\textit{medium}).
\item[Statement contracts.]
No more supported since \textsf{Frama-C 23} (Vanadium) because of
unsoundness bugs to be fixed and \textsf{ACSL} restrictions.
Support shall be restored on a mid-term basis (\textit{easy}).
\item[Non-natural loops.]
Loop constructed with \textsf{goto} are no more supported since
\textsf{Frama-C 23} (Vanadium) because of unsoundness bugs to be fixed.
A new engine is under construction but is not yet ready (\textit{medium}).
\item[Dynamic allocation.]
All implemented memory models \emph{are} able to deal with dynamic allocation,
which is actually used internally to manage the scope of local variables.
However, \textsf{ACSL} clauses for specifying allocation and deallocation
are not implemented yet (\textit{medium}).
\item[Assigns.]
The WP strategy for proving assign clauses is a based on a sound but incomplete verification: we check that side effects
(writes and called assigns) are \emph{included} in specified assigns, which
is a sufficient but not necessary condition. The known workaround is to
specify larger assigns and to add the missing equalities to contracts.
Indeed, looking for an efficient and more permissive strategy would be challenging (\textit{hard}).
\item[Froms.]
Proving \textsf{ACSL} assigns-from clauses is not implemented. It is as difficult as proving functional assigns. Although, we have designed some
heuristics to prove assigns-from clauses in simple cases that could be implemented on a mid-term basis (\textit{medium}).
\item[Per-behavior assigns.]
Different assigns clause associated with distinct behaviors are difficult to
take into account without a deep refactoring of the WP rule of calls. We currently use a sound upper-approximation of assigns for function calls that might make correct \textsf{ACSL} properties not provable by lack of precision
(\textit{medium}).
\item[Bytes, unions \& casts.]
Memory models with non-typed access and bit- or byte-precision access would
be easy to implement, although terribly inefficient. We are currently working on a new memory analysis that would provide a deep understanding of how to make \emph{different} memory models soundly working with each others on distinct memory regions. This would deserve a brand new research plan, to be founded by collaborative projects (\textit{hard}).
\item[Floats.]
A new sound but incomplete model for floats is provided since \textsf{Frama-C 21} (Scandium). Here, by incomplete we means that it is generally difficult
to prove arithmetic properties of float operations by lack of good support from \textsf{SMT} solvers. Although, recent advances on our \textsf{Colibri} solver open the road to a better support for float operations in a near future (\textit{medium/hard}).
\item[Function pointers.]
Limited support for function pointers is provided \emph{via} an extension of \textsf{ACSL}, see Section~\ref{acsl:calls} for details. Currently, a function
pointer must be provably equal to a finite set of known functions. Although,
this could be easily extended to support function contracts refinement (\textit{medium}).
\end{description}
