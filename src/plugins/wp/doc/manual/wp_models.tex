\chapter{WP Models}
\label{wp-models}

Basically, a memory model is a set of datatypes, operations and
properties that are used to abstract the values living inside the heap
during the program execution.

Each memory model defines its own representation of pointers, memory
and data actually stored in the memory. The memory models also define
some types, functions and properties required to translate \textsf{C}
programs and \textsf{ACSL} annotations into first order logic
formul{\ae}.

The interest of developing several memory models is to manage the
trade-off between the precision of the heap's values representation
and the difficulty of discharging the generated proof obligations by
external decision procedures. If you choose a very accurate and
detailed memory model, you shall be able to generate proof obligations
for any program and annotations, but most of them would not be
discharged by state-of-the art external provers. On the other hand,
for most \textsf{C} programs, simplified models are applicable and
will generate less complex proof obligations that are easier to
discharge.

A practical methodology is to use the simpler models whenever it is
possible, and to up the ante with more involved models on the remaining, more
complex parts of the code.

This chapter is dedicated to the description of the memory models
implemented by the \textsf{WP} plug-in.  In this manual, we only
provide a high-level description of the memory models you might select
with option \texttt{-wp-model} (section~\ref{wp-model-logical}
and~\ref{wp-model-pointers}). Then we focus on two general powerful
optimizations. The first one, activated by default
(section~\ref{wp-model-logicvar}), mixes the selected memory model
with the purely logical Hoare model for those parts of your program
that never manipulate pointers. The second one
(section~\ref{wp-model-byreference}) is dedicated to those pointers
that are formal parameters of function passed by reference.

\section{Language of Proof Obligations}
\label{wp-lang}

The work of \textsf{WP} consists in translating \textsf{C} and
\textsf{ACSL} constructs into first order logical formul{\ae}. We
denote by $\cal{L}$ the logic language for constructing proof
obligations.  Shortly, this logical language is made of terms
($t:\mathrm{term}$) and propositions ($P:\mathrm{prop}$) that consist of:
\begin{itemize}
\item Natural, signed, unbounded integer constants and their operations;
\item Natural real numbers and their operations;
\item Arrays (as total maps) and records (tuples with named fields);
\item Abstract (polymorphic) data types;
\item Anonymous function symbols with (optional) algebraic properties;
\item Logical connectors;
\item Universally and existentially quantified variables.
\end{itemize}

Actually, the task of the memory model consists in mapping any
heap \textsf{C}-values at a given program point to some variable or term
in the logical $\cal{L}$ language.

\section{The Hoare Memory Model}
\label{wp-model-logical}

This is the simplest model, inspired by the historical definition of
\emph{Weakest Precondition Calculus} for programs with no pointers. In
such programs, each global and local variable is assigned a
distinct variable in $\cal{L}$.

Consider for instance the statement \lstinline{x++;} where
\lstinline{x} has been declared as an \lstinline{int}. In the
\lstinline{Hoare} memory model, this \textsf{C}-variable will be
assigned to two $\cal{L}$-variables, say $x_1$ before the statement, and
$x_2$ after the statement, with the obvious relation $x_2 = x_1+1$ (if
no overflow occurred).

Of course, this model is not capable of handling memory reads or writes
through pointer values, because there is no way of representing
aliasing.

You select this memory model in the \textsf{WP} plug-in with the option
\texttt{-wp-model Hoare}; the analyzer will complain whenever you
attempt to access memory through pointers with this model.

\section{Memory Models with Pointers}
\label{wp-model-pointers}

Realistic memory models must deal with reads and writes to memory
through pointers. However, there are many ways for modeling the raw
bit stream the heap consists of. All memory models $\cal{M}$ actually
implement a common signature:
\begin{description}
\item[Pointer Type:] $\tau$, generally a pair of a base address and an offset.
\item[Heap Variables:] for each program point, there is a set of
  logical variables to model the heap. For instance, you may have a
  variable for the values at a given address, and another one for the
  allocation table. The heap variables $m_1\ldots m_k$ are
  denoted by $\overline{m}$.
\item[Read Operation:] given the heap variables $\overline{m}$, a
  pointer value $p:\tau$, and some \textsf{C}-type $T$, the
  model will define an operation:
  \[\mathrm{read}_T(\overline{m},p) : \mathrm{term}\]
  that defines the representation in $\cal{L}$ of the value of
  \textsf{C}-type $T$ which is stored at address $p$ in the heap.
\item[Write Operation:] given the heap variables $\overline{m}$ before
  a statement, and their associated heap variables $\overline{m}'$
  after the statement, a pointer value $p:\tau$ and a value $v$ of
  \textsf{C}-type $T$, the model will define a relation:
  \[\mathrm{write}_T(\overline{m},p,v,\overline{m}') : \mathrm{prop}\] that relates the
  heap before and after writing value $v$ at address $p$ in the heap.
\end{description}

Typically, consider the statement \lstinline{(*p)++} where
\lstinline{p} is a \textsf{C}-variable of type \lstinline{(int*)}.
The memory model $\cal{M}$ will assign a unique pointer value
$P:\tau$ to the address of \lstinline{p} in memory. 

Then, it retrieves the actual value of the pointer
$\lstinline{p}$, say $A_p$, by reading a value of type
\lstinline{int*} into the memory variables $\overline{m}$ at address
$P$:
\[ A_p = \mathrm{read}_{\mathtt{int*}}(\overline{m},P) \]

Next, the model retrieves the previous \lstinline{int}-value at
actual address $A_p$, say $V_p$:
\[ V_p = \mathrm{read}_{\mathtt{int}}(\overline{m},A_p) \]

Finally, the model relates the final memory state $\overline{m}'$
with the incremented value $V_p+1$ at address $P$:
\[ \mathrm{write}_{\mathtt{int}}(\overline{m},A_p,V_p+1,\overline{m}') \]

\section{Hoare Variables mixed with Pointers}
\label{wp-model-logicvar}

As illustrated above, a very simple statement is generally translated
by memory models into complex formul{\ae}. However, it is possible in
some situations to mix the Hoare memory model with the other ones.

For instance, assume the address of variable \lstinline{x} is never
taken in the program. Hence, it is not possible to create a pointer
aliased with \lstinline{&x}. It is thus legal to manage the value of
\lstinline{x} with the Hoare memory model, and other values with
another memory-model $\cal{M}$ that deals with pointers.

Common occurrences of such a situation are pointer variables. For
instance, assume \lstinline{p} is a variable of type \lstinline{int*};
it is often the case that the value of \lstinline{p} is used (as in
\lstinline{*p}), but not the address of the variable \lstinline{p}
itself, namely \lstinline{&p}. Then, it is very efficient to manage
the value of \lstinline{p} with the Hoare memory model, and the value
of \lstinline{*p} with a memory model with pointers.

Such an optimization is possible whenever the address of a variable is
never taken in the program. It is activated by default in the
\textsf{WP} plug-in, since it is very effective in practice. You can
nevertheless deactivate it with selector ``\texttt{-wp-model raw}''.

\section{Hoare Variables for Reference Parameters}
\label{wp-model-byreference}

A common programming pattern in \textsf{C} programs is to use pointers
for function arguments passed by reference. For instance, consider the
\lstinline{swap} function below:

\begin{ccode}
void swap(int *a,int *b)
{
  int tmp = *a ;
  *a = *b ;
  *b = tmp ;
}
\end{ccode}

Since neither the address of \lstinline{a} nor the one of \lstinline{b}
are taken, their values can be managed by the Hoare Model as described
in the previous section. But we can do even better. Remark that none of
the pointer values contained in variables \lstinline{a} and
\lstinline{b} is stored in memory. The only occurrences of these
pointer values are in expressions \lstinline{*a} and
\lstinline{*b}. Thus, there can be no alias with these pointer values
elsewhere in memory, provided they are not aliased initially.

Hence, not only can \lstinline{a} and \lstinline{b} be managed by the
Hoare model, but we can also treat \lstinline{(*a)} and \lstinline{(*b)}
expressions as two independent variables of type \lstinline{int} with
the Hoare memory model.

For the callers of the \lstinline{swap} function, we can also take benefit
from such by-reference passing arguments. Typically, consider the
following caller program:

\begin{ccode}
void f(void)
{
  int x=1,y=2 ;
  swap(&x,&y);
}
\end{ccode}

Strictly speaking, this program takes the addresses of \lstinline{x}
and \lstinline{y}. Thus, it would be natural to handle those variables
by a model with pointers. However, \lstinline{swap} will actually
always use \lstinline{*&x} and \lstinline{*&y}, which are respectively
\lstinline{x} and \lstinline{y}.

In such a situation it is then correct to handle those variables
with the Hoare model, and this is a very effective optimization in
practice. Notice however, that in the example above, the optimization
is only correct because \lstinline{x} and \lstinline{y} have disjoint
addresses.

These optimizations can be activated in the \textsf{WP} plug-in with
selector ``\texttt{-wp-model ref}'', and the necessary separation
conditions are generated on-the-fly. This memory model first detects
pointer or array variables that are always passed by reference. The
detected variables are then assigned to the Hoare memory model.

This optimization is not activated by default, since the non-aliasing
assumptions at call sites are sometimes irrelevant.

\section{Mixed models hypotheses}
\label{wp-model-hypotheses}

For the previously presented \textsf{Ref} model, but also for the
\textsf{Typed}, and \textsf{Caveat} models presented later, WP lists
the separation and validity hypotheses associated to the choice that
it make of dispatching each pointer and variable either to the
\lstinline{Hoare} or to the model $\cal{M}$ used for the heap.

Consequently, in addition to user-defined function \lstinline{requires},
WP also assumes, and thus states as \lstinline{requires} for the function
caller, that:

\begin{itemize}
\item \lstinline{Hoare} variables are separated from each other,
\item \lstinline{Hoare} variables are separated from the locations in $\cal{M}$,
\item references are valid memory locations,
\item locations assigned via a pointer are separated from \lstinline{Hoare}
  variables whose address is not taken by the function (including via
  its contract).
\end{itemize}

Furthermore, the function must ensure that:

\begin{itemize}
\item locations assigned via a pointer (including the returned value when
  it is a pointer) are separated from \lstinline{Hoare} variables whose address
  is not taken by the function (including via its contract),
\item pointers assigned by the function (including the returned value when
  it is a pointer) are separated from function parameters and \lstinline{Hoare}
  variables whose address is not taken by the function (including via
  its contract).
\end{itemize}

In order to precisely generate these hypotheses, WP needs precise
\lstinline{assigns} specification. In particular each function under
verification and all its callees needs an \lstinline{assigns} specification.
Furthermore, if the function assigns or returns a pointer, WP needs
a correct \lstinline{\from} specification for those pointers. If the
specification is incomplete, a warning \lstinline{wp:pedantic-assigns} is
triggered. Note that WP does not verify that the \lstinline{\from} is correct.

The hypotheses are displayed when the option
\lstinline{-wp-warn-memory-model} is enable (it is enabled by default).
They can be verified by WP using the experimental option
\lstinline{-wp-check-memory-model}.


\section{The Typed Memory Model}

This memory model is actually a reformulation of the \texttt{Store}
memory model used in previous versions of the \textsf{WP} plug-in. In
theory, its power of expression is equivalent. However, in practice,
the reformulation we performed makes better usage of built-in theories
of \textsf{Alt-Ergo} theorem prover and \textsf{Coq} features.
The main modifications concern the heap encoding and the
representation of addresses.

\paragraph{Addresses.} We now use native records of $\cal{L}$ and provers 
to encode addresses as pairs of base and offset (integers). This
greatly simplifies reasoning about pointer separation and commutation of
memory accesses and updates.

\paragraph{Store Memory.} In the \texttt{Store} memory model, the heap is 
represented by one single memory variable holding an array of
\emph{data} indexed by \emph{addresses}. Then, integers, floats and
pointers must be boxed into \emph{data} and unboxed from \emph{data}
to implement read and write operations. These boxing-unboxing
operations typically prevent \textsf{Alt-Ergo} from making maximal usage
of its native array theory.

\paragraph{Typed Memory.} In the \textsf{Typed} memory model, the heap is 
now represented by \emph{three} memory variables, holding respectively
arrays of integers, floats and addresses indexed by addresses. This
way, all boxing and unboxing operations are avoided. Moreover, the
native array theory of \textsf{Alt-Ergo} works very well with its
record native theory used for addresses: memory variables
access-update commutation can now rely on record theory to decide that
two addresses are different (separated).

\section{The Caveat Memory Model}
\label{CAVEAT}

This memory model simulates the behavior of the \textsf{Caveat}
analyser, with additional enhancements.  It is implemented as an
extension of the \texttt{Typed} memory model, with a specific
treatment of \textit{formal} variables of pointer type.

To activate
this model, simply use '\verb$-wp-model Typed+Caveat$' or
'\verb$-wp-model Caveat$' for short.

A specific detection of variables to be treated as \textit{reference
  parameters} is used.  This detection is more clever than the standard one
since it only takes into account local usage of each function (not
global ones).

%However, it can not be applied with axiomatics that
%read into the memory. Only pure logic functions and pure predicates can be
%defined in axiomatics with this model.

Additionally, the \texttt{Caveat} memory model of \textsf{Frama-C}
performs a local allocation of formal parameters with pointer
types that cannot be treated as \textit{reference parameters}.
Note that it means that the pointer is considered valid. If one needs
to accept \texttt{NULL} for this pointer, a \texttt{wp\_nullable} ghost
annotation or the clause \texttt{wp\_nullage\_args} can be used to bring
this information to WP:

\listingname{nullable.c}
\cinput{nullable.c}

This makes explicit the separation hypothesis of memory regions
pointed by formals in the \textsf{Caveat} tool. The major benefit of
the \textsf{WP} version is that aliases are taken into account by the
\texttt{Typed} memory model, hence, there are no more suspicious
\textit{aliasing} warnings.

\paragraph{Warning:} using the \texttt{Caveat} memory model,
the user \emph{must} check by manual code review that no aliases are
introduced \emph{via} pointers passed to formal parameters at call sites.

However, \textsf{WP} warns about the implicit separation hypotheses required by
the memory model \textit{via} the \texttt{-wp-warn-memory-model} option, set
by default.


%% For the benefit of the clarity, we note $\cal{L}$ the logic language
%% used by the rules of the \textsf{WP} calculus.  Proof obligations are
%% expressed into that language, and \textsf{ACSL} annotations are
%% translated into terms of $\cal{L}$.

%% The next section describes the translation of the \textsf{ACSL} part
%% which is independent of the memory model. The translation of
%% the other part of \textsf{ACSL} is described into specific sections
%% for each kind of memory model.

% vim: spell spelllang=en

