
\section{The Typed Memory Model}

This memory model is actually a reformulation of the \texttt{Store}
memory model used in previous versions of the \textsf{WP} plug-in. In
theory, its power of expression is equivalent. However, in practice,
the reformulation we performed makes better usage of built-in theories
of \textsf{Alt-Ergo} theorem prover and \textsf{Coq} features.
The main modifications concern the heap encoding and the
representation of addresses.

\paragraph{Addresses.} We now use native records of $\cal{L}$ and provers 
to encode addresses as pairs of base and offset (integers). This
greatly simplifies reasoning about pointer separation and commutation of
memory accesses and updates.

\paragraph{Store Memory.} In the \texttt{Store} memory model, the heap is 
represented by one single memory variable holding an array of
\emph{data} indexed by \emph{addresses}. Then, integers, floats and
pointers must be boxed into \emph{data} and unboxed from \emph{data}
to implement read and write operations. These boxing-unboxing
operations typically prevent \textsf{Alt-Ergo} from making maximal usage
of its native array theory.

\paragraph{Typed Memory.} In the \textsf{Typed} memory model, the heap is 
now represented by \emph{three} memory variables, holding respectively
arrays of integers, floats and addresses indexed by addresses. This
way, all boxing and unboxing operations are avoided. Moreover, the
native array theory of \textsf{Alt-Ergo} works very well with its
record native theory used for addresses: memory variables
access-update commutation can now rely on record theory to decide that
two addresses are different (separated).
