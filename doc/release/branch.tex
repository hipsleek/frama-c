\chapter{Branch stage}
\label{chap:branch}

That is the procedure for forking the release from \texttt{master}.

\section{Creating the milestones}

Create the milestone for the next releases on \textsf{Gitlab},
in the Frama-C group. They will be used for development that will not
be integrated into the upcoming release.
\expertise{François, Julien}.

\section{``Freezing'' master}

When the ``freeze'' period arrives (usually a few weeks before the beta
release), the release manager may decide to prevent direct merges to the
\texttt{master} branch, until the \texttt{stable} branch is created.
If so, then the branch should be protected such that Developers can no
longer push and merge to it, without asking for a Master to do it.

In Gitlab, this is done via: Frama-C project $\rightarrow$ Settings
$\rightarrow$ Repository $\rightarrow$ Protected Branches.

\textbf{The \texttt{master} branch must be protected \emph{at all times}.
  Parameters
  \texttt{Allowed to merge} and \texttt{Allowed to push} must be set to
  \texttt{Developers + Maintainers} during non-freeze times, and to
  \texttt{Maintainers} during freeze time.}

\section{Creating the branch}

Note that you must be member of the GitLab groups "frama-c", "dev" and have
access to each plugin of continuous integration.

Create the branch \texttt{stable/release}, where \texttt{release} is the
element name, using the \texttt{frama-ci} tool:
\begin{enumerate}
\item Install \texttt{frama-ci} tools:
\begin{shell}
opam pin add frama-ci-tools git@git.frama-c.com:frama-c/Frama-CI.git
\end{shell}
\item Create an API token for gitlab, in your gitlab profile settings.
\item Run the command
\begin{shell}
frama-ci-create-branch --token=\$TOKEN \
--name=stable/release --default-branch
\end{shell}
This command creates a branch \texttt{stable/release} for frama-c and for
each plugin tested by the CI — and configures the CI to use these branches
by default when testing fixes for the release.
These branches are directly created on the gitlab server.
\end{enumerate}
What can be committed in this branch must follow the release schedule,
and go through Merge-requests. Everything else should go in \texttt{master},
which can then be reset to standard-level protection (Developers + Maintainers
allowed to push/merge).

\section{GitLab issues}

{\em This is currently done periodically in specific Frama-C meetings, so only
  a final check is usually necessary.}~\\

Check public issue tracker at \url{https://git.frama-c.com/pub/frama-c/issues/}.
All issues should have been at least acknowledged: at least they should be
assigned to someone, preferably tagged appropriately.

Send a message to the Frama-C channel on LSL's Mattermost. Each developer should
have a look at their own assigned/monitored/reported issues. Depending on their
severity, the issues should be tagged with the current release, or with the next
one.

\section{Version}

On the new \texttt{stable} branch, execute the script:
\begin{verbatim}
./dev/set-version.sh NN.M # to be replaced with actual major/minor version
\end{verbatim}
This will:
\begin{itemize}
  \item update the \texttt{Changelog}s
  \item update the changes in the manuals \textbf{(excluding ACSL and E-ACSL references)}
  \item update the \texttt{VERSION} and \texttt{VERSION\_CODENAME} files
  \item update the \texttt{opam} files (Frama-C, lint, hdrck)
  \item update the API doc
  \item update the Frama-C build script
  \item update the reference configuration
\end{itemize}

Merge the \texttt{stable} branch in the \texttt{master} branch.

On the \texttt{master} branch, execute the script:
\begin{verbatim}
  ./dev/set-version.sh dev
\end{verbatim}
This will:
\begin{itemize}
  \item update the \texttt{VERSION} file
  \item update the \texttt{opam} files (Frama-C, lint, hdrck)
\end{itemize}

Commit this change and push.

\section{Copyright}

Check that the date in copyright headers is correct. If not then:
\begin{itemize}
  \item update the dates in the license files in:
  \begin{itemize}
    \item \texttt{headers/closed-source/*}
    \item \texttt{headers/open-source/*}
    \item \texttt{ivette/headers/closed-source/*}
    \item \texttt{ivette/headers/open-source/*}
    \item \texttt{src/plugins/e-acsl/headers/closed-source/*}
    \item \texttt{src/plugins/e-acsl/headers/open-source/*}
  \end{itemize}
  \item update the headers with the following command:
  \begin{itemize}
    \item \texttt{make headers}
  \end{itemize}
  \item Check if some copyrights are left to update by \texttt{grep}-ing the date in the sources: \texttt{grep -r -e "-yyyy" .}
\end{itemize}

%%%Local Variables:
%%%TeX-master: "release"
%%%End:
