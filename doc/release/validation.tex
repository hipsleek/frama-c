\chapter{Validation stage}
\label{chap:validation}

Final validation before running the release pipeline.

\section{Prerequisites on dependencies}

Install all dependencies of Frama-C, including recommended ones.

Check coherence between:
\begin{itemize}
  \item \texttt{opam} \todo{Should always be up to date}
  \item \texttt{reference-configuration.md} \todo{Should always be up to date}
\end{itemize}

Run the command:
\begin{itemize}
  \item \texttt{dune build @frama-c-configure}
\end{itemize}
all plugins should be enabled. Check dependencies in case they have changed.
If you don't know them, ask plug-in developers to verify them.

\section{Prerequisites on installation}

\begin{itemize}
  \item Check the contents of \texttt{INSTALL.md} \todo{Should always be up to date}
  \item Check the contents of \texttt{README.md} \todo{Should always be up to date}
\end{itemize}

\section{Fix version}

The tasks listed in this section are performed by the Release Manager.

\subsection{Versions in documentation files}

Change version and codename in the following files:
\begin{itemize}
  \item \texttt{ALL\_VERSIONS} (non-beta only)
  \item \texttt{VERSION} (for beta releases, add suffix \texttt{\textasciitilde{}beta}, not \texttt{-beta})
  \item \texttt{opam}
  \begin{itemize}
     \item change version (for beta releases, add suffix \texttt{\textasciitilde{}beta}, not \texttt{-beta})
     \item change doc link: \texttt{…/download/user-manual-<version>-<codename>.pdf}
           (with \texttt{-beta} in version for beta releases, not \texttt{\textasciitilde{}beta})
  \end{itemize}
\end{itemize}

\subsection{Versions in source: API documentation}

Check that no \verb|+dev| suffix remains inside comments:

\begin{shell}
git grep +dev src
\end{shell}

Test by recompiling Frama-C and rebuilding the API documentation:
\begin{shell}
dune build @install
dune build @doc
\end{shell}

You can also check that the elements referred to in the index of the
development guide are marked as such in the API doc and vice-versa
by issuing \texttt{make check-devguide}\todo{Fix this}.

\subsection{Manuals}

Manuals are built by the continuous integration. However, the Release Manager
should check whether a new version of ACSL or E-ACSL must be released.

Also, most of the manuals include an appendix with the list of changes that have
happened, divided by version of Frama-C. Make sure that the first section has
the correct release name.

Manuals experts:
\begin{center}
  \begin{tabular}{lll}
    \hline
    \textsf{User Manual}     & \texttt{doc/userman}   & \expertise{Julien} \\
    \textsf{Developer Guide} & \texttt{doc/developer} & \expertise{Julien} \\
    \textsf{ACSL}            & (from Github)          & \expertise{Virgile} \\
    \textsf{Aoraï}           & \texttt{doc/aorai}     & \expertise{Virgile} \\
    \textsf{E-ACSL} & \texttt{src/plugins/e-acsl/doc/*} & \expertise{Julien} \\
    \textsf{Eva}             & \texttt{doc/eva}       & \expertise{David} \\
    \textsf{Metrics}         & \texttt{doc/metrics}   & \expertise{André} \\
    \textsf{RTE}             & \texttt{doc/rte}       & \expertise{Julien} \\
    \textsf{WP}  & \texttt{src/plugins/wp/doc/manual/} & \expertise{Loïc} \\
  \end{tabular}
\end{center}

\subsection{Contributors}

Update the Frama-C's authors list in files
\texttt{src/plugins/gui/help\_manager.ml} and \texttt{opam}. Traditionally,
an author is added if they have contributed 100+ LoC in the release, as counted
by:
\begin{verbatim}
git ls-files -z | \
parallel -0 -n1 git blame --line-porcelain | \
sed -n 's/^author //p' | sort -f | uniq -ic | sort -nr
\end{verbatim}
(source for the command: \url{https://gist.github.com/amitchhajer/4461043})

\subsection{Commit}

Commit any change that you have done during these checks \textbf{and push}.

\section{Last pipeline artifacts validation}

In the last continuous integration pipeline of the release branch, force the
run of the following targets:
\begin{itemize}
  \item manuals
  \item opam-pin
  \item opam-pin-minimal
\end{itemize}
They shall succeed. Collect the artifacts of the following targets:
\begin{itemize}
  \item api-doc
  \item build-distrib-tarball
  \item manuals
\end{itemize}

Check that these artifacts are as expected. In particular:
\begin{itemize}
  \item API documentatation:
    \begin{itemize}
      \item check that you can browse the API documentation
      \item if there are minor bugs/typos, note them for later, but it's not
        worth delaying the release to fix them.
    \end{itemize}
  \item Check versions in manuals
  \item Tarball
    \begin{itemize}
      \item Check that no non-free components are distributed
      \item Check that no \texttt{/home/user} path can be found in the distribution,
      \item Build and test
        \begin{itemize}
          \item on Linux (the CI has already done it, but the GUI is not tested)
          \item on MacOS \expertise{André, Loïc}
          \item on Windows (WSL) \expertise{Allan, André}
        \end{itemize}
      \item consider decompressing both the current and the previous release
         archives in different directories, then using \texttt{meld} to compare
         them. This allows spotting if some files were unexpectedly added, for
         instance.
    \end{itemize}
\end{itemize}

Alternatively, you can use \texttt{docker} to compile the archive against a
precise configuration:
\begin{itemize}
  \item \verb+cp distributed/frama-c-<VERSION>.tar.gz developer_tools/docker+
  \item \verb+cd developer_tools/docker+
  \item \verb+make Dockerfile.dev+
  \item \verb+docker build . -t framac/frama-c:dev --target frama-c-gui-slim \+\\
        \verb+  -f Dockerfile.dev --build-arg=from_archive=frama-c-<VERSION>.tar.gz+
\end{itemize}
For the GUI: in order to be able to launch
\verb+x11docker framac/frama-c:dev frama-c-gui+,
you might want to install the
\href{https://github.com/mviereck/x11docker}{\texttt{x11docker}} script.

\section{Validate release}

Create the main changes file in the directory \texttt{releases}. This file must
be named <VERSION without ext>.md (e.g. \texttt{25.0}: \texttt{25.0.md},
\texttt{25.0~beta}: \texttt{25.0.md}, \texttt{25.1}: \texttt{25.1.md}). The
expected format is:

\begin{lstlisting}
  # Kernel

  - item
  - ...

  # <Plugin-Name> (Prefer alphabetic order)

  - item
  - ...

  # ...
\end{lstlisting}

It should only list main changes, that will be displayed on the event section
of the website and the wiki page.

Create the version commit, tag it using \texttt{git tag \$(cat VERSION | sed -e "s/\textasciitilde /-/")}
and push it (e.g. via \texttt{git push origin \$(cat VERSION | sed -e "s/\textasciitilde/-/")}).
\textbf{
  If the tagged commit itself has not been pushed, remember to push it, else,
  the release pipeline will fail.
}

Note that an extra pipeline will be run when the tag is pushed, and it will
likely fail due to external plugins. This is not a problem (if you really want
to avoid it, you can create and push the tag to each external plug-in).

%%%Local Variables:
%%%TeX-master: "release"
%%%End:
