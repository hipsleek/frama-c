\chapter{Retrouver l'information}\label{sec-find}

Jusqu'à présent, on a vu comment calculer le graphe de dépendances.
Dans ce chapitre, nous présentons les fonctions fournies pour
trouver des éléments du PDG suivant un certain nombre de critères.
L'annexe \ref{sec-impact} rappelle certains objectifs initiaux
d'une utilisation de ces fonctions pour effectuer une analyse d'impact.\\

Par ailleurs, outre la manipulation d'ensemble d'éléments du PDG,
un système de gestion et de propagation de marques est également proposé.
Il est présenté au chapitre \ref{sec-mark}.

\section{A partir de leur clé}

Au cours du calcul, chaque élément du PDG est associé à une clé
qui correspond à l'élément de programme qu'il représente (cf. {\tt
pdgIndex.mli}).
On peut par la suite retrouver un élément à partir de cette clé
comme par exemple l'élément correspondant à une instruction simple,
à un paramètre d'entrée, à une déclaration locale, etc.

\section{A partir d'une zone mémoire}

Grâce à l'état qui est propagé lors de la construction
des dépendances de données (cf. \S\ref{sec-propagation-etat})
on peut retrouver les éléments qui participent au calcul
d'une zone mémoire à un point de programme.

Cette fonction doit également vérifier si les éléments trouvés
définissent complètement la donnée recherchée ou non,
et si ce n'est pas le cas, indiquer la zone susceptible de ne pas être
complètement définie au point considéré.

\section{A partir de propriétés}

D'autres fonctionnalités de \ppc permettent d'interpréter une propriété
du programme et d'en extraire les zones mémoire nécessaires à son évaluation.
Comme on sait par ailleurs trouver les éléments qui correspondent
a une zone mémoire en un point de programme (voir ci-dessus),
on peut ainsi trouver par exemple les éléments dont dépendent une assertion
ou tout autre propriété.

\section{En exploitant les dépendances}

Après avoir retrouver des éléments dans le graphe, on veut en général exploiter
leurs dépendances. On peut facilement retrouver les dépendances avant ou
arrière,
avec ou sans filtrage sur leur type (donnée, contrôle,...),
récursivement ou non.
