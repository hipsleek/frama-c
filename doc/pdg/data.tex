\chapter{Dépendances liées aux données}

Le calcul de dépendances de données et d'adresse consiste principalement
à retrouver les éléments de flot correspondant aux données utilisées dans les
expressions.
Mais comme les données peuvent être incluses les unes dans les
autres, il ne suffit pas de retrouver les éléments de flot qui calculent
exactement ces données, mais aussi ceux qui ont une intersection possible.

Par exemple, dans la séquence~:\\
  \centerline{\verbtt{d = d0; x = d.a;}}

il faut être capable de voir que \verbtt{x} dépend de \verbtt{d0}.

Autre exemple~: dans la séquence~:\\
\centerline{\verbtt{d0.a = a; d0.b = b; d = d0;}}

il faut voir que \verbtt{d} dépend éventuellement de la valeur initiale de \verbtt{d0}
(si \verbtt{d0} contient d'autres champs que \verbtt{.a} et \verbtt{.b}), mais aussi
de \verbtt{a}, et de \verbtt{b}.\\

\section{Recherche arrière}

Le premier calcul mis en {\oe}uvre procédait par recherche en arrière
des éléments de la table ayant une intersection avec les données présentes en
partie droite de l'instruction. Mais cette recherche était compliquée en cas de
dépendances partielles comme dans le second exemple ci-dessus. Cette solution a
donc été abandonnée.

\section{Propagation avant d'un état}\label{sec-propagation-etat}

La méthode finalement choisie pour calculer ces dépendances
consiste à propager en avant, par une analyse du type flot de données,
un \indexdef{état des données}
qui contient pour chaque donnée, une liste de liens vers les éléments
du graphe qui ont permis de déterminer sa valeur en ce point.

Cet état doit avoir les propriétés suivantes~:
\begin{itemize}
\item on veut pouvoir associer un nouveau noeud à une donnée
en précisant s'il faut faire l'union avec l'ensemble précédemment stocké.
Par exemple~;
\begin{itemize}
\item pour l'instruction \verbtt{x = 3;} on construit un nouvel élément
dans le PDG, et on mémorise dans l'état que \verbtt{x} est maintenant associé
à cet élément. L'ancienne association est perdue.
\item  pour l'instruction \verbtt{*p = 3;} si \verbtt{p} peut pointer sur \verbtt{x},
il faut mémoriser que  \verbtt{x} peut être défini par l'élément du PDG
correspondant, mais comme ce n'est pas sûr, il faut faire l'union avec
ce qui était précédemment stocké pour \verbtt{x}.
\end{itemize}
\item lorsque l'on demande l'ensemble associé à une donnée,
le résultat doit contenir au moins ce qu'on a stocké
(il peut contenir plus d'élément en cas de perte de précision),
\item la consultation ne doit pas modifier l'état,
\item il faut savoir faire l'union de deux états.
\end{itemize}

La structure de donnée du module \verbtt{Lmap} de \ppc correspond à ces critères,
et peut donc être utilisée pour ce calcul.\\

\section{Propagation arrière d'un état}

Une autre solution aurait pu être de propager en arrière
un état contenant les utilisations
de variables, et mettre à jour le graphe en rencontrant la définition.
Le coût de ce calcul semble être le même que le précèdent,
mais la propagation avant nous permet d'avoir, à chaque point de programme,
un état donnant la correspondance entre une donnée et les éléments du graphe,
même si cette donnée n'est pas utilisée à ce point.
Cette information nous permet par la suite de définir des critères
de \slicing moins restrictifs.

\section{Traitement de l'affectation}

Le principe de l'algorithme du traitement d'une affectation
\verbtt{lval = exp;} est donc le suivant~:

\begin{itemize}
\item recherche des données $\{d_v\}$
utilisées dans \verbtt{exp} à l'aide des résultats de l'analyse d'alias préalable,
\item calcul de {\it dpdv},
c'est-à-dire l'union des ensembles associées à ces données $d_v$ dans l'état,
\item recherche des données $\{d_a\}$
utilisées pour calculer l'adresse de {\it lval},
\item calcul de {\it dpda}
c'est-à-dire l'union des ensembles associées à ces données $d_a$ dans l'état,
\item recherche de l'élément $e$
correspondant à cette instruction dans le graphe,
      et création de cet élément s'il n'existe pas,
\item ajout des dépendances {\it dpdv} et  {\it dpda} à $e$,
\item recherche des données $\{d_x\}$
potentiellement modifiées par cette affectation,
\item calcul du nouvel état (après l'instruction)
en ajoutant dans l'ancien état un lien entre les $\{d_x\}$ et $e$.
\end{itemize}

\section{Déclarations}

Les déclarations de variable doivent être traitées séparément
des valeurs, car on peut parfois dépendre de l'adresse d'une variable
sans dépendre de ce qu'elle contient.

C'est par exemple le cas lorsque la variable apparaît à gauche
d'une affectation (\verbtt{x = 3;})
ou encore quand on n'utilise que son adresse (\verbtt{p = \&x;}).

On garde donc une table qui permet de retrouver les éléments
du graphe de dépendances qui correspondent aux déclarations.


\section{Calcul de conditions}

Les noeuds du graphe de dépendances représentant
les calculs de condition des \verbtt{if} ou \verbtt{switch}
ont des dépendances de donnée sur les données utilisées.

\section{Dépendances de donnée dans les boucles}

Pour les boucles {\it explicites}, il suffirait d'effectuer deux tours
de la boucle pour obtenir toutes les dépendances de donnée car le premier
capture les dépendances avec les données provenant d'avant la boucle,
ou interne à un tour, et le second capture les dépendances entre un tour et le
suivant.

Mais comme les boucles peuvent être introduites par la présence de sauts
quelconques,
le plus simple est dans un premier temps d'itérer jusqu'à obtenir un point fixe
sur l'état des données. Il faut noter que les noeuds ne sont créés que lors de la
première itération, les suivantes n'ajoutant que de nouvelles dépendances entre
ces noeuds. Le nombre de noeuds étant fini (de l'ordre de grandeur du nombre
d'instruction de la fonction), l'atteignabilité du point fixe est
garantie.

\section{Appels de fonction}

Le traitement des appels de fonction est présenté
dans le chapitre \ref{sec-intro-call}
