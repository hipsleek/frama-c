%==================================================================================
\chapter{Slicing}

\section{Ligne de commande}

Parmi les options utilisables depuis la ligne de commande 
 qui sont spécifiques au greffon de {\sl slicing}, trois options permettent
 de spécifier les modes de fonctionnement du greffon :

\begin{description}
\item {\tt -slice-print} :   
      {\sl pretty print the sliced code to the standard out, or else
        into the file specified with the {\tt -ocode} option.}

\item {\tt -slice-undef-functions} : 
      {\sl allow the use of the -slicing-level option for calls to
        undefined functions (by default, don't slice the prototype of
        undefined functions).}

\item {\tt -slicing-level n} :  
      {\sl set the default level of slicing used to propagate to the calls.}
      {\sl The possible values for {\tt n} are :}
  \begin{description}
  \item {\tt 0} :  
        {\sl don't slice the called functions.}
  \item {\tt 1} :   
        {\sl don't slice the called functions but propagate the marks anyway.}
  \item {\tt 2} :   
        {\sl try to use existing slices, create at most one.}
  \item {\tt 3} :   
        {\sl most precise slices.}
  \end{description}
  {\sl note: this value (defaults to 2) is not used for calls to
    undefined functions except when {\tt-slice-undef-functions} option is set.}


\end{description}

Les options suivantes permettent de spécifier les réductions à
effectuer en précisant ce en quoi l'utilisateur est intéressé~:
\begin{description} 
\item {\tt -slice-assert f1,...,fn} :  
      {\sl select the assertions of functions {\tt f1},...,{\tt fn}.}

\item {\tt -slice-calls f1,...,fn} :  
      {\sl select every calls to functions {\tt f1},...,{\tt fn}, and all their dependencies.}

\item {\tt-slice-loop-inv f1,...,fn} : 
      {\sl select the loop invariants of functions {\tt f1},...,{\tt fn}.}
\item {\tt -slice-loop-var f1,...,fn} : 
      {\sl select the loop variants of functions {\tt f1},...,{\tt fn}.}
\item {\tt -slice-pragma f1,...,fn} :  
      {\sl use the slicing pragmas in the code of functions {\tt f1},...,{\tt fn} as slicing criteria.}

\item {\tt -slice-return f1,...,fn} :  
      {\sl select the result (returned value) of functions {\tt f1},...,{\tt fn}.}

\item {\tt -slice-threat f1,...,fn} :  
      {\sl select the threats of functions {\tt f1},...,{\tt fn}.}

\item {\tt -slice-value v1,...vn} :  
      {\sl select the result of left-values {\tt v1},...,{\tt vn}
        at the end of the function given as entry point
        (the addresses are evaluated at the beginning of the function given as entry point).}

\item {\tt -slice-rd v1,...vn} :  
      {\sl select the read access to left-values {\tt v1},...,{\tt vn}
        (the addresses are evaluated at the beginning of the function given as entry point).}

\item {\tt -slice-wr v1,...vn} :  
      {\sl select the write access to left-values {\tt v1},...,{\tt vn}
        (the addresses are evaluated at the beginning of the function given as entry point).}

\end{description}

Ces options ne sont pas exclusives 
car il est possible de s'intéresser simultanément 
à plusieurs fonctions 
ainsi qu'à plusieurs aspects concernant une même fonction.
Une même option peut donc être utilisée à plusieurs reprise.
Notons que l'ordre n'intervient pas dans le résultat final.

\section{Annotation pour le slicing}

Les annotations destinées à spécifier des requêtes
de slicing se situe, comme toutes les autres annotations, dans des commentaires
C commençant par \verb!@!. Leur identification est \verb!slice pragma!,
et elles sont prises en compte si l'option \verb!-slice-pragma! mentionnée
ci-dessus est utilisée.\\

Il existe trois requêtes de slicing~:
\begin{quote}
\begin{description}
  \item[]\verb!//@ slice pragma expr! {\it expr\_desc} \verb!;!
    correspond au critère traditionnel de slicing, à savoir préserver le passage
    au point de programme et la valeur de l'expression à ce point. Exemple~:
    \begin{center}
    \verb!//@slice pragma expr S.a;!
    \end{center}
  \item[]\verb!//@ slice pragma ctrl;!
    permet de s'intéresser uniquement au passage à un point de programme 
    (sélection des dépendances de contrôle),
  \item[]\verb!//@ slice pragma stmt;!
    sélectionne l'instruction (et donc de tout ce qui permet de l'atteindre et
    de la calculer).
    Cette dernière requête est expérimentale.
\end{description}
\end{quote}





