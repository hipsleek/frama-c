
\chapter{Exemple de marquage interprocédural}

Cette partie présente comment les actions élémentaires précédemment présentées
peuvent être utilisées pour répondre à des requêtes utilisateur de plus haut
niveau.


\section{Présentation de l'exemple}

Nous allons voir différents exemple de marquage de l'exemple ci-dessous.
Dans tous les cas, on considère que l'utilisateur souhaite n'avoir qu'une
spécialisation par fonction source dans le résultat, et qu'il demande
systématiquement la propagation de son marquage aux appelants.
\bb

Pour simplifier la présentation,
comme on ne s'intéresse ici qu'à la propagation interprocédurale,
on n'utilise qu'une marque élémentaire $m$ quelconque qui rend un élément
visible et la marque \spare{} déjà mentionnée.

\noindent\begin{tabular}{p{4.4cm}p{4.5cm}p{5.3cm}}
\begin{verbatim}
int X, Y;
int g (int u, int v,
       int w) {
  lg1: X = u;
  lg2: Y = u + v;
  return w;
  }
\end{verbatim}
&
\begin{verbatim}
int Z;
int f (int a, int b,
       int c, int d,
       int e) {
  int r;
  lf1: r = g (a,b,c);
  lf2: Z = g (r,d,e);
  return X;
}
\end{verbatim}
&
\begin{verbatim}
int I, J, K, L, M, N;
int main () {
  lm1: /* ... */
  lm2: N = f (I,J,K,L,M);
  lm3: /* ... */
}
\end{verbatim}
\end{tabular}

\bb\centerline{\uneimage{exple2v0}}\bb

La légende indique comment sont représentées les marques sur les figures
suivantes.

\section{Cas 1}

Supposons tout d'abord que l'utilisateur veuille sélectionner la sortie 0
de $f$, et voyons comment se déroule le calcul.\bb

On a vu en \S\ref{sec-propagate-to-calls} que cette requête de l'utilisateur se
traduit par la séquence d'actions élémentaires suivantes~:
\begin{itemize}
  \item $f_1 = \actNewSlice (f_0)$,
  \item $\actAddOutputMarks(f_1, (out_0, m))$
  \item ${\mathit main}_1 = \actNewSlice ({\mathit main}_0)$,
  \item $\actChangeCall(c, {\mathit main}_1, f_1)$.
\end{itemize}

On calcule tout d'abord le marquage de $f_1$ par simple propagation~:

\bb{\centerline{\uneimage{exple2v1-1}}}\bb

Puis, la seconde action
génère $\actChooseCall(c_1, f_1)$ et $\actChooseCall(c_2, f_1)$.

A l'application de la première de ces nouvelles actions,
comme il n'y a pas encore de spécialisation pour $g$,
on génère~:

\begin{itemize}
  \item $g_1 = \actNewSlice (g_0)$,
  \item $\actAddOutputMarks(g_1, (out_0, m))$
  \item $\actChangeCall(c_1, f_1, g_1)$
\end{itemize}

A l'issue de la construction de $g_1$, l'entrée $w$ a une marque $m_2$.
L'application du \actChangeCall{}
va donc déclencher un $\actModifCallInputs(c_1, f_1)$ qui va conduire à marquer
l'entrée $c$ de $f_1$ comme $\spare$ (en plus de $m$ en $m_1$).

\bb{\centerline{\uneimage{exple2v1-2}}}\bb

L'application de $\actChooseCall(c_2, f_1)$ produit~:
\begin{itemize}
  \item $\actAddOutputMarks(g_1, (out_X, m))$ puisqu'on a choisi de n'avoir
    qu'une spécialisation par fonction source,
  \item et $\actChangeCall(c_2, f_1, g_1)$.
\end{itemize}

Comme $g_1$ est appelée en $c_1$,
la modification de son marquage conduit à générer $\actMissingInputs(c_1, f_1)$,
qui, vues les options, est directement traduit par
$\actModifCallInputs(c_1, f_1)$.

Puis, le $\actChangeCall{}$ va déclencher $\actModifCallInputs(c_2, f_1)$
qui va marquer $e$ comme $\spare$.

\bb{\centerline{\uneimage{exple2v1-3}}}\bb

Finalement, il ne reste plus qu'à appliquer $\actChangeCall(c, {\mathit main}_1,
f_1)$ qui conduit à appliquer $\actModifCallInputs(c, {\mathit main}_1)$
et donc à propager le marquage de $f_1$ dans ${\mathit main}_1$.

\bb{\centerline{\uneimage{exple2v1-4}}}


\section{Cas 2}

A partir du résultat du cas 1, l'utilisateur souhaite sélectionner le calcul de
$Y$ dans $g_1$. \bb

$\actAddUserMark(g_1, (out_Y, m))$ propage le marquage en $m_1$ à l'entrée $v$
de $g_1$ ($u$ est déjà marquée).

\bb{\centerline{\uneimage{exple2v2-1}}}\bb

Puis comme $g_1$ est appelée, et qu'il manque des marques, deux actions
$\actMissingInputs{}$ sont générées. Les options indiquent qu'elles doivent être
traduites en $\actModifCallInputs(c_1, f_1)$ et $\actModifCallInputs(c_2, f_1)$.
Ce qui conduit à marquer en $m_1$ les entrées $a, b, d$ de $f_1$.

\bb{\centerline{\uneimage{exple2v2-2}}}\bb

De même, la propagation va être effectuée dans ${\mathit main}_1$
par l'application de $\actModifCallInputs(c, {\mathit main}_1)$.

\bb{\centerline{\uneimage{exple2v2-3}}}\bb


\section{Cas 3}

Dans une nouvelle étude, l'utilisateur souhaite sélectionner le calcul de $X$
dans $g$.
\bb

Comme dans le cas 1, cette requête se traduit par la création d'une fonction
spécialisée $g_1$ et la propagation de son marquage à tous ses appels~:
\begin{itemize}
  \item $g_1 = \actNewSlice (g_0)$,
  \item $\actAddOutputMarks(g_1, (out_X, m))$
  \item $f_1 = \actNewSlice (f_0)$,
  \item $\actChangeCall(c_1, f_1, g_1)$.
  \item $\actChangeCall(c_2, f_1, g_1)$.
  \item ${\mathit main}_1 = \actNewSlice ({\mathit main}_0)$,
  \item $\actChangeCall(c, {\mathit main}_1, f_1)$.
\end{itemize}

On calcule tout d'abord le marquage de $g_1$~:

\bb{\centerline{\uneimage{exple2v3-1}}}\bb

Puis, les deux \actChangeCall{} dans $f_1$ conduisent à propager $m_1$~:

\bb{\centerline{\uneimage{exple2v3-2}}}\bb

La seconde propagation (lors de la modification de $c_2$)
déclenche $\actMissingOutputs(c_1, f_1)$
qui, au vue des options, se transforme en $\actAddOutputMarks(g_1,
\outsigc(c_1))$. Ceci conduit à marquer $m_2$ la sortie 0 de $g_1$.
Les $\actMissingInputs$ générés se transforme en $\actModifCallInputs(c_1, f_1)$
et $\actModifCallInputs(c_2, f_1)$, qui propage \spare{} aux entrées $c$ et $e$ de
$f_1$.

\bb{\centerline{\uneimage{exple2v3-3}}}\bb

Enfin, le $\actChangeCall(c, {\mathit main}_1, f_1)$ propage le marquage de
$f_1$ dans ${\mathit main}_1$. On remarque que même si ce changement avait été
effectué plus tôt, la propagation du marquage aurait été effectué grâce à des
$\actMissingInputs$.

\bb{\centerline{\uneimage{exple2v3-4}}}\bb


\section{Cas 4}

A partir du cas 3, l'utilisateur souhaite ajouter la sélection du calcul de $Y$
dans $g_1$.\bb

$\actAddUserMark(g_1, (out_Y, m))$ conduit à marquer $Y$ en $m_1$,
puis, par propagation, $v$ en $m_1$ également~:

\bb{\centerline{\uneimage{exple2v4-1}}}\bb

Des actions $\actMissingInputs$ transformée en $\actModifCallInputs$ propage
$m_1$ aux entrées $b$ et $d$ de $f_1$, puis aux entrées $J$ et $L$ de ${\mathit
main}_1$.

\bb{\centerline{\uneimage{exple2v4-2}}}\bb
