\chapter{Projets}\label{sec-projets}

Ce chapitre regroupe diverses idées qui ont été évoquées à un moment ou un
autre, et qui ont soit été abandonnées, soit remises à plus tard.

\section{Autres critères}

\subsection{Spécifier ce qu'on ne veut plus}

Un autre critère intéressant de \slicing, qui n'apparaît dans aucun papier,
serait, je pense, de pouvoir spécifier ce que l'on veut enlever.
Par exemple, {\it "que devient le programme si on ne s'intéresse pas
à telle sortie ou à tel traitement"}.\\

Par ailleurs, l'analyse de dépendance utilisée pour le \slicing
peut également être utilisée pour calculer des informations utiles à d'autres
analyses. On peut par exemple s'intéresser aux variables dont la valeur
détermine la sortie d'une boucle; ce qui peut être utiliser comme un indice
sur les élargissements à effectuer dans une interprétation abstraite.

\subsection{Calcul d'une variable globale}

Le premier filtrage global vise à sélectionner les instructions qui
participent au calcul d'une variable globale donnée.
Comme on a accès par ailleurs à la liste des sorties de chaque fonction,
il suffit de générer les filtres permettant de calculer
cette variable sur chaque fonction l'ayant comme sortie.

\section{Utilisation de la sémantique}

Cette partie regroupe quelques idées d'évolutions pour améliorer le \slicing
d'une fonction en utilisant la sémantique des instructions.
Certaines d'entre elles ont été implémentées dans l'outil précédent;
d'autres ne sont que des pistes qui restent à explorer.

\subsection{Réduction relative à une contrainte}

Dans l'outil précédent, une commande permettait de couper une branche
en remplaçant le test de la condition de chemin par un {\tt assert}.\\

Plus généralement,
on s'intéresse ici à une contrainte représentant une assertion dans le code.
On aimerait déplacer cette contrainte
afin de déterminer un point où elle est impossible (réduite à {\it false}).
On pourrait alors {\it couper} alors la branche correspondante.

Le test conditionnel
permettant d'accéder à cette branche peut être transformé en assertion,
ou être supprimé (option ?).
Dans le premier cas, les données utilisées dans le test
sont visibles alors que dans le second,
elles peuvent éventuellement être supprimées
si elles ne servent pas par ailleurs.

On peut aussi ôter les instructions qui ne sont plus utiles.
La détection de code mort permet par exemple de supprimer les éléments qui
n'étaient utilisés que dans la branche supprimée.

\begin{exemple}
Dans la séquence suivante~:
\begin{center}
\mbox{\texttt x = c+1; a = 2; if (c) x += a; y = x+1;}
\end{center}

si on demande à sélectionner les dépendances de \verbtt{y}, puis à supprimer la
branche \verb!if (c)!, on obtient~:
\begin{center}
\mbox{\texttt x = c+1; \sout{a = 2;} {\bf assert (!c);} \sout{x += a;} y = x+1;}
\end{center}

où l'on voit que l'instruction \verb!a=2;! disparaît puis qu'elle ne sert plus.\\

On verra ci-dessous comment on peut aussi déterminer que \verbtt{x} vaut alors $1$
à l'aide du calcul de WP, et que donc, \verbtt{y} vaut 2 à l'aide de la
propagation de constantes.
\end{exemple}

Le problème général lorsque l'on impose des contraintes,
c'est qu'elles peuvent aussi influencer les résultats d'autres analyses
comme l'analyse de valeur. On en verra un cas particulier
ci-dessous avec la propagation de constante.

%--------------------------------------
\subsection{Passage à un point de programme}

La contrainte dont nous venons de parler peut également être donnée
implicitement en spécifiant un point de programme. Il s'agit alors d'être
capable de supprimer ce qui ne sert pas lorsque l'exécution de la fonction passe
par ce point. Une telle requête peut par exemple être utilisée dans le cadre
d'une analyse d'impact.\\

Une première étape peut consister à supprimer les branches incompatibles avec le
point en ne regardant que la syntaxe.

\begin{exemple}
Dans la séquence suivante~:
\begin{center}
\mbox{\texttt if (c) P: x += a; else y = x+1;}
\end{center}
si l'utilisateur souhaite examiner le passage au
point P, on peut éliminer la branche {\texttt else}~:
\begin{center}
\mbox{\texttt {\bf assert (c);} P: x += a; \sout{else y = x+1;}}
\end{center}
Attention, ceci n'est vrai que si la séquence n'est pas dans une boucle...
\end{exemple}

Mais il peut être plus intéressant de calculer une précondition qui assure que
l'on passe par le point donné, et d'utiliser ce résultat comme une contrainte.\\

\remarque{rem-wp} Attention,
il faut noter que cette précondition ne peut généralement pas être
calculée par un simple WP car celui-ci assure qu'une propriété est vraie
si la fonction est exécutée avec des entrées satisfaisant la précondition,
mais il n'assure pas qu'il n'y a pas d'autres cas possibles.
Ceci est du à l'approximation introduite par le calcul de WP des boucles.

%--------------------------------------
\subsection{Propagation de constantes}

D'autres réductions peuvent être effectuées en exploitant
les constantes du programme.

\begin{exemple}
Dans la séquence suivante~:
\begin{center}
  \verb!x = c ? 1 : -1; if (x<0) f(x); else g(x); !
\end{center}
si l'on étudie cette séquence avec une valeur initiale 0 pour \verbtt{c}
on aimerait savoir déterminer que :
\begin{itemize}
\item \verb!x=-1!,
\item donc que le second test est vrai,
\item que donc \verbtt{g} ne sera pas appelé,
\item et que \verbtt{f} sera appelé dans ce contexte avec la valeur \verb!-1!.\\
\end{itemize}

On souhaite donc obtenir~:
\begin{center}
  \verb!assert (c == 0); x = -1; f(-1); !
\end{center}
\end{exemple}

Ce calcul peut facilement être effectué par une analyse de valeur externe,
mais il faut pouvoir lui transmettre la contrainte (ici~: $c=0;$).\\


Dans l'outil précédent, le module de \slicing (FLD) avait été couplé à
un module de propagation de constante (CST) de la façon suivante~:
\begin{itemize}
  \item pour chaque fonction filtrée, on calcule les états CST correspondants,
  \item si une branche est supprimée à la demande de l'utilisateur,
    CST en est informé pour réduire l'état,
  \item si une constante est associée à une donnée,
    (par exemple lors de la spécialisation d'une fonction avec un paramètre
constant)
    CST en est également informé,
  \item lors du calcul d'une fonction filtrée, pour chaque saut conditionnel,
    on interroge CST pour savoir si une branche est inaccessible ($\bot$),
    et si c'est le cas, les éléments correspondant sont marqués $mM$ (code
    mort),
  \item pour chaque appel de fonction, si l'on sait déterminer qu'une ou
    plusieurs entrées sont constantes, on demande la spécialisation de cette
    fonction,
  \item pour chaque appel de fonction, si l'on sait déterminer qu'une ou
    plusieurs sorties sont constantes, on fournit l'information à CST pour qu'il
    puisse la propager.
\end{itemize}

\begin{exemple}
  Reprenons l'exemple précédent où l'on sait que $c=0$ en 1~:
\begin{center}
  \verb!/*1*/ x = c ? /*2*/1 : /*3*/-1; /*4*/if (x<0) /*5*/f(x); else /*6*/g(x); !
\end{center}
On a donc~:
\begin{itemize}
  \item en \verb!/*1*/! : $c=0$,
  \item en \verb!/*2*/! : $\bot$ puisque le test est faux,
  \item en \verb!/*4*/! : $x=-1$,
  \item en \verb!/*5*/! : toujours $x=-1$,
  \item en \verb!/*6*/! : $\bot$ puisque le test est toujours vrai,
\end{itemize}
Cette séquence est donc réduite à~:
\begin{alltt}
  x = \sout{ c ? 1 : } -1; \sout{ if (x>0)} f\_1(x); \sout{ else  g(x); }
\end{alltt}
où \verbtt{f\_1} est la fonction \verbtt{f} spécialisée avec son entrée à -1.\\
\end{exemple}

Quand l'analyse permet de déterminer la valeur précise d'une variable,
il peut également être intéressant d'utiliser cette information pour effectuer
une transformation de programme juste avant de produire le résultat.
On ne peut pas vraiment le faire avant, car on peut être amené à regrouper
plusieurs fonctions où la valeur peut être différente.\\


L'implémentation des points suivants n'a pas été fait dans l'outil précédent,
et il faudrait donc voir s'il est possible de les intégrer
dans le nouveau~:
  \begin{itemize}
    \item supprimer l'argument correspondant à une entrée constante.
En effet, lors du calcul de la fonction spécialisée, on ne savait
pas supprimer l'argument correspondant,
mais on propageait tout de même la valeur pour supprimer les éventuelles
branchesmortes.
\begin{exemple}
Dans l'exemple précédent,
on peut spécialiser \verbtt{f} en \verbtt{f\_1} car on sait que $x=-1$.

\begin{tabular}{ccl}
Mais on aimerait aussi transformer &:& \verb! void f_1 (int x) { ... }!\\
en &:& \verb! void f_1 (void) { int x = -1; ... }!
\end{tabular}
\end{exemple}
\item supprimer le calcul d'une condition de \verbtt{if} toujours vraie quand il
  n'y a pas de \verbtt{else}. Cela n'avait pas été fait car dans
le module de propagation de constante,
il n'y avait pas de point de programme correspondant à la branche
\verbtt{else} manquante. L'état $\bot$ qui lui aurait été
 associé n'était donc pas là pour permettre cette suppression.

  \end{itemize}

%--------------------------------------
\subsection{Utilisation de WP}

Comme on l'a vu, la propagation de constante peut permettre de
spécialiser les fonctions et de couper des branches de programme.

Ce type calcul s'effectue a priori par propagation avant.
Pour faire de la propagation arrière,
nous aimerions utiliser un calcul de WP.

Le principe d'utilisation serait le suivant~:

\begin{exemple1}
  L'utilisateur s'intéresse uniquement à la branche {\it then} du {\it if}.
  On introduit donc l'assertion $y<0$~:\\

\begin{center}
\begin{footnotesize}
\begin{tabular}{|p{5cm}|p{5cm}|}
  \hline
Programme source & coupure de branche \\
  \hline
\begin{verbatim}
int f (int c, int x) {

  int y = c ?  1 : -1;
  if (y < 0) {
    int z = y - 1;
    g (z);
    return z + x;
    }
  else
    return 0;
}
\end{verbatim}
&
\begin{alltt}
int f (int c, int x) \{

  int y = c ?  1 : -1;
 {\bf assert (y < 0); }\{
  int z = y - 1;
  g (z);
  return z + x;
  \}


\}
\end{alltt}
\\
\hline
\end{tabular}
\end{footnotesize}
\end{center}
\end{exemple1}
\begin{exemple2}
\begin{center}
\begin{footnotesize}
\begin{tabular}{|p{5cm}|p{5cm}|}
\hline
WP & Propagation de constante \\
\hline
\begin{alltt}
int f (int c, int x) \{
 {\bf /* (c == 0); */ }
  int y = c ?  1 : -1;
  assert (y < 0); \{
    int z = y - 1;
    g (z);
    return z + x;
  \}

\}
\end{alltt}
&
\begin{alltt}
int f (int c, int x) \{
  /* (c == 0);  */
  int y = {\bf /* 0 ?  1 : */} -1;
  assert ({\bf -1} < 0); \{
    int z = {\bf -1} - 1;
    g ({\bf -2});
    return {\bf -2} + x;
  \}

\}
\end{alltt}
\\
\hline
\end{tabular}
\end{footnotesize}
\end{center}

Sur cet exemple, l'utilisation du WP permet de déterminer que $c$ vaut 0,
et qu'il est alors possible de propager cette constante.\\

\end{exemple2}


La principale difficulté de l'utilisation d'un tel calcul est le pilotage des
opérations.

Le plus simple est de laisser l'utilisateur contrôler la propagation
en utilisant des commandes interactives.

L'idée est de lui permettre de donner une contrainte,
de la remonter à l'aide de WP, et d'en déduire d'autres propriétés,
utilisable par exemple par l'analyse de valeur.\\

Attention, ici encore, comme dans la remarque \vref{rem-wp},
il faudra préalablement
s'assurer que l'approximation introduite par le WP des boucles est
bien dans le bon sens relativement à ce que l'on souhaite obtenir...
