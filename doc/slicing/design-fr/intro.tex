\chapter{Introduction}


L'objectif de ces travaux est de concevoir et développer un outil de \slicing
se basant sur le noyau \ppc d'analyse statique de programme C
en s'inspirant du module équivalent déjà développé
sur la base de l'outil \caveat qui est décrit dans \cite{slicingCaveat}.



\section{État de l'art}\label{sec-lart}

Voyons tout d'abord
ce que dit la littérature sur ce sujet
afin de se positionner par rapport aux différentes méthodes.\\

On s'appuie ici sur la présentation des travaux sur le calcul de PDG
effectué dans un autre document.

\subsection{Origine}

La technique consistant à réduire un logiciel -~appelée \slicing en anglais~-
a été beaucoup
étudiée depuis son introduction par la thèse de \cite{weiser79}.
L'article \cite{weiser81} rappelle que sa définition du \slicing est
la suivante~:

\begin{definition}{slicing selon Weiser}
Le programme d'origine et le programme transformé doivent être identique
si on les regarde à travers la fenêtre défini par le critère de \slicing
défini par un point de programme et une liste de variable.
C'est à dire que la séquence de valeurs observées pour ces variables
à ce point doivent être les mêmes.

De plus, le programme transformé doit être obtenu par suppression de certaines
instructions du programme d'origine.
\end{definition}

Cette définition a été remise en cause par la suite
au motif que cette définition n'est pas suffisante pour traduire
la compréhension intuitive que l'on a du \slicing.
On trouve l'exemple suivant dans \cite{kumar02better},
mais cette critique revient a plusieurs reprises.

\begin{exemple}
\begin{center}
\begin{tabular}{|l|l|l|}
\hline
P & P1 & P2 \\
\hline
a = 1;   &          & a = 1; \\
b = 3;   &          & b = 3; \\
c = a*b; &          & \\
a = 2;   & a = 2;   & \\
b = 2;   & b = 2;   & \\
d = a+b; & d = a+b; & d = a+b; \\
\hline
\end{tabular}
\end{center}
P1 est probablement ce qu'on attend du \slicing de P si l'on s'intéresse
à la variable d après la dernière instruction, mais P2
est aussi correct du point de vue de Weiser.
\end{exemple}

Il semble peu probable de construire un algorithme qui construit P2,
mais on voit néanmoins que la définition est insuffisante
pour spécifier complètement ce que doit être le résultat d'un \slicing.\\

\cite{Choi94} formalise la définition de la façon suivante~:

\begin{definition}{Correct-Executable-Slice}
Let $P_{slice}$ be an executable slice of a program $P_{org}$,
S be the set of statements in $P_{slice}$ and $Proj_S(P_{org}, \sigma_0)$
be the projection of $Trace(P_{org}, \sigma_0)$ that contains
only those statement traces produced by statements in S.
$P_{slice}$ is a correct executable slice of $P_{org}$ if
 $Proj_S(P_{org}, \sigma_0)$  is identical to $Trace(P_{org}, \sigma_0)$.
\end{definition}

Même si cette définition possède les mêmes inconvénients que celle d'origine,
elle lui permet de faire des preuves de correction des ses algorithmes
relativement à cette formalisation.

\subsection{Graphes de dépendances}

Comme on l'a vu dans \cite{ppcPdg} les graphes de dépendances sont à la base des
calculs.
\cite{Ottenstein84}, puis \\
\cite{Ferrante87}
introduisent la notion de PDG ({\it Program Dependence Graph}).
Un tel graphe est utilisé pour représenter les différentes dépendances
entre les instructions d'un programme.
Ils l'exploitent pour calculer un {\it slicing non-executable}
qui s'intéresse uniquement aux instructions
qui influencent la valeur des variables,
mais cette représentation sera également utilisée pour calculer un
\slicing exécutable.\\

Cette représentation, initialement intraprocédurale, a été étendue
à l'interprocédurale dans \cite{horwitz88interprocedural} où elle
porte le nom de SDG ({\it System Dependance Graph}).
Dans ce papier, Susan Horwitz s'intéresse à un \slicing
ayant la définition suivante~:

\begin{definition}{slicing selon \cite{horwitz88interprocedural}}
A slice of a program with respect to program point p and variable x
consists of a set of statements of the program that might affect
the value of x at p.
\end{definition}

Mais elle précise par ailleurs que x doit être défini ou utilisé en p.
En effet, lorsque le graphe est construit, le \slicing se résume à un
problème d'accessibilité à un noeud, or un noeud représente un
point de programme et ne contient que les relations concernant les variables
qui y apparaissent.\\

Nous avons vu dans \cite{ppcPdg} que l'on peut lever cette limitation en gardant
une correspondance entre les données à un point de programme et les noeud du
graphe.

\subsection{Slices vs. Chops}

On a vu que le \slicing s'intéresse à ce qui se passe avant d'arriver
à un certain point.\\

La notion de {\it chopping} est similaire sauf qu'elle s'intéresse
aux instructions qui vont être influencées par une certaine donnée
à un certain point.

\subsection{Critères de \slicing}

Nous n'avons pour l'instant parlé que du \slicing dit {\bf statique}
qui s'intéresse à toutes les exécutions d'un programme,
par opposition au \slicingb {\bf dynamique}
 qui considère la projection du programme
sur l'exécution d'un jeu d'entrée particulier,
ou {\bf quasi-statique} qui fixe uniquement une partie des entrées.\\

Une version plus générale qui inclus à la fois l'un et l'autre
est appelé \slicingb {\bf conditionnel} (\cite{fox-consit}).
Il s'agit de spécifier le cas d'étude sous forme de
des propriétés sur le jeu d'entrées (préconditions).\\

Par la suite, la notion de {\bf  \slicing conditionnel arrière}
a été introduit dans plusieurs buts différents :
\begin{itemize}
\item la première technique décrite dans \cite{Comuzzi} et
 appelée p-slice, s'intéresse aux instructions qui participent
à l'établissement de la postcondition,
\item alors que la deuxième (\cite{fox01backward})
a un objectif un peu différent car vue
comme une aide à la preuve permettant d'éliminer du programme
tous les cas dans lesquels la propriété est vérifiée (automatiquement),
pour ne garder que ceux pour lesquels le résultat n'est pas certain
afin que l'utilisateur puisse se concentrer sur la vérification
de cette partie.

Dans \cite{Harman2001}, ces auteurs continuent
les travaux précédents en combinant
une propagation avant et une propagation arrière.
Quelques exemples et la présentation d'un outil mettant en {\oe}uvre
le travail de cette équipe est présenté dans \cite{daoudi-consus}.
Ils définissent leur objectif de la façon suivante :
\end{itemize}

\begin{definition}{pre-post conditioned slicing}
Statements are removed if they cannot lead to
satisfaction of the negation of the post condition,
when executed in an initial state which satisfies
the pre-condition.
\end{definition}

\begin{itemize}
\item une troisième approche est présentée très brièvement
dans (\cite{Chung-2001}~:
il s'agit également de spécifier le critère de slicing par une pré-post
et de combiner une propagation avant et arrière de propriétés
pour ne garder que ce qui participe
à l'établissement de la post dans le contexte de la précondition.
\end{itemize}


% {\bf contraint} (\cite{Field95}),


Ces nouveaux types de \slicing
ne se basent plus uniquement sur des relations de dépendance,
mais nécessitent différentes techniques complémentaires comme
l'exécution symbolique ou encore le calcul de WP.\\

Dans le cadre de notre étude, il semble intéressant d'explorer
cette voie étant donné que nous disposons (ou disposerons)
de tels outils d'analyse dans \ppc.\\







\subsection{Transformations}

Dans nos précédents travaux, nous avons exploré certaines transformations
de programme comme la spécialisation de fonction
(et donc la transformation des appels),
et également la propagation de constantes.\\

Divers papiers, dont \cite{harman01gustt},
présentent un \slicing qu'ils appellent {\bf amorphe}
dans lequel il s'agit de
combiner des techniques de \slicing et de transformation de programme.

 \cite{ward2002} (pour une raison qui m'échappe)
préfère parler de  \slicingb {\bf sémantique}\\

% {\bf à développer !}


\subsection{Outils}

On peut citer deux outils qui mettent en {\oe}uvre cette technique sur les
programmes C-ANSI~:
\begin{itemize}
\item {\it unravel}, présenté dans \cite{lyle97using},
  est disponible sur le web  (cf. [\cite{unravel}])~;
\item et le {\it Wisconsin Program-Slicing Tool} (\cite{reps93wisconsin}),
est commercialisé par {\it GrammaTech} sous le nom {\it CodeSurfer}
(cf. site web [\cite{CodeSurfeur}]).\\
\end{itemize}

Ces outils permettent tous les deux~:
\begin{itemize}
\item de sélectionner les parties de code utiles au calcul
d'une variable à un point de contrôle,
\item de visualiser les instructions sélectionnées sur le
code source de l'utilisateur,
\item et de faire des opérations d'union et l'intersection entre plusieurs
réductions.
\end{itemize}

\subsection{Pour en savoir plus}

Un état de l'art en 1995 est présenté dans \cite{tip95survey}.\\
Plus récemment, \cite{survey2005} font un tour très complet des évolutions
jusqu'en 2005.

Par ailleurs, un site internet~- dont l'adresse est donnée
dans la bibliographie sous la référence \cite{wwwSlicing}~-
a tenu à jour une liste des projets
et une bibliographie très impressionnante sur le \slicing
jusqu'en 2003, mais ce site ne semble plus maintenu.\\


\section{Ce que l'on veux faire}\label{sec-but}\label{sec-defR}

\newcommand{\remitem}[1]{\begin{itemize}\item #1 \end{itemize}}
%\newcommand{\warnitem}[1]{\begin{itemize}\item[$\blacktriangle$] #1
                          %\end{itemize}}
%\newcommand{\question}[1]{\begin{itemize}\item[{\bf ?}] #1 \end{itemize}}
%\newcommand{\smilitem}[1]{\begin{itemize}\item[$\smiley$] #1 \end{itemize}}
%\newcommand{\frownitem}[1]{\begin{itemize}\item[$\frownie$] #1 \end{itemize}}

Le document \cite{baudinImpact} spécifie un outil d'aide à l'analyse d'impact
après avoir analysé les besoins dans ce domaine.
Le module de \slicing est présenté comme étant
un composant important de cet outil.\\

Résumons ici les caractéristiques principales de l'outil
qui ont servies de spécification initiale~:

\begin{itemize}
\item l'outil sert à construire un programme compilable
à partir d'un programme source, les deux programmes ayant le même
comportement vis à vis des critères fournis par l'utilisateur~;
\item la construction de ce programme résultat est un processus interactif.
\item {\bf commandes} de l'outil
\remitem{elles correspondent à des fonctions internes (en \caml)
et doivent permettre d'accéder aux différentes fonctionnalités.\\
Elles pourront être utilisées directement,
combinées dans des {\it scripts} pour effectuer des opérations
plus évoluées ou encore servir à construire un protocole de communication avec
une interface graphique~;}
\item commandes {\bf d'interrogation}
\remitem{elles servent à questionner le système.
Elles ne doivent pas modifier l'état interne de l'outil.
Même si elles effectuent certains calculs pour répondre,
et qu'elles stockent ces résultats pour éviter un recalcul ultérieur,
ceci doit être transparent pour la suite des traitements~;}
\item commandes dites {\bf d'action}
\remitem{elles doivent permettent de construire
le résultat du \slicing. Ces commandes doivent pouvoir s'enchaîner et se
combiner~;}
\item l'outil maintient une {\bf liste d'attente} des actions à effectuer.
Ceci est utile car certains traitements peuvent être décomposés
en plusieurs actions que l'utilisateur peut ensuite choisir d'appliquer ou non.
Cela permet d'avoir un contrôle plus fin des calculs~;
\item le programme résultant ne peut être généré
que lorsque la liste d'attente est vide~;
\item les traitements doivent être modulaires~;
\item chaque fonction du programme source peut~:
  \begin{itemize}
  \item ne pas apparaître dans le résultat,
  \item être simplement copiée dans le résultat,
  \item correspondre à une ou plusieurs fonctions spécialisées du résultat.
  \end{itemize}
\item dans la représentation interne du résultat, les instructions
sont associées à un marquage qui indique si elles doivent être cachées,
ou sinon, la raison de leur présence, c'est-à-dire par exemple
si elles participent au contrôle ou à la valeur d'une donnée sélectionnée, etc.

\item identification d'instructions non significatives (ou superflues)
\remitem{il s'agit d'être capable de distinguer ce que l'on doit ajouter
lorsque l'on ne fait pas de spécialisation.
Prenons par exemple le cas d'un appel $L : f(a,b);$
où seul $a$ est nécessaire au calcul qui nous intéresse,
mais que l'on ne souhaite pas spécialiser en $f(a);$.
Il faudra alors ajouter ce qui permet de calculer $b$ en $L$
aux instructions sélectionnées,
mais en préservant l'information qu'elles sont {\it superflues}~;
}

\item spécialisation du code
\remitem{il s'agit d'être capable d'effectuer des transformations syntaxiques
pour réduire le code au maximum. Par exemple, si après une instruction
\verbtt{y = x++;}, on n'a besoin que de \verbtt{x},
il faut être capable de décomposer cette instruction
afin de ne garder que \verbtt{x++;}

Ce type de transformation étant déjà faite en amont du module de \slicing,
le problème ne se pose plus pour ce type d'instruction,
mais il existe encore pour la spécialisation des appels de fonction.
En effet, si une fonction $f$ prend, par exemple,
deux arguments dont un seul sert à calculer ce qui nous intéresse,
on veut être capable de construire une fonction $f_1$
qui ne prend qu'un seul argument,
et transformer l'appel à $f$ en un appel à $f_1$.

Par ailleurs, on peut également envisager de transformer des
instructions lorsque l'on connaît la valeur constante d'une variable.
}
\end{itemize}


\section{Plan des documents}

Ce document présente uniquement les principes de ce qui a été développé.
Pour plus de détail sur l'implémentation ou les commandes disponibles,
le lecteur est invité à consulter la documentation du code
car elle sera toujours plus détaillée et à jour que ce rapport.
Comme la boîte à outil de \slicing est un greffon de \ppc,
son interface est définie dans le module \verbtt{Slicing}.
Par ailleurs, l'introduction de la documentation interne du module
donne une idée de l'architecture, et donne des points d'entrée.\\

Les principes de calcul des graphes de dépendances
sur lequel s'appuie le \slicing est présenté dans \cite{ppcPdg}.\\

Le chapitre \ref{sec-filtrage-fct} présente
la façon dont sont marquées les instructions d'une fonction pour
effectuer une réduction intraprocédurale.\\

Puis, le chapitre \ref{sec-interproc} expose comment sont organisés les calculs
pour faire une propagation des réductions à l'ensemble de l'application,
et comment on se propose de spécialiser les fonctions.\\

Enfin, le chapitre \ref{sec-man} présente comment utiliser le module
développé et propose donc une vision davantage dédiée à un utilisateur
potentiel.\\


Ce rapport étant un document de travail, il va évoluer au fur et à mesure des
réalisations~:
\begin{itemize}
  \item la version 1.0 du 27 juin 2006 présentait
les développements en cours concernant les graphe de dépendance
et le marquage d'une fonction. La présentation des travaux à faire
pour la gestion d'un \slicing interprocédural était moins détaillée car
plus prospective. Suivait un chapitre contenant un manuel d'utilisation.
Enfin, la conclusion présentait l'état courant de l'outil,
et les perspectives d'évolution.
\item dans la version 2.0 du 26 juin 2007, la partie sur le graphe de
  dépendance a été extraite car elle fait l'objet d'un document séparé.
  Le \slicing interprocédural est cette fois présenté en détail.
  Les parties issues du document de spécification dont les idées
  n'ont finalement pas été retenues ont été déplacées en annexe
  \ref{sec-projets} pour mémoire. La partie sur l'utilisation de l'outil
  a pour l'instant été conservée, même si elle n'est pas très spécifique à
  l'utilisation du greffon de \slicing.
\end{itemize}

Fin 2008, ce document peut être considéré comme stable,
car même si le développement évolue, il ne s'agit plus de modifier les
fonctionnalité de bases, mais plutôt d'ajuster des points de détail pour
améliorer la précision par exemple.
