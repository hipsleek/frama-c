\chapter{Réduction d'une fonction}\label{sec-filtrage-fct}

\section{Fonction spécialisée}

On appelle \indextxtdef{fonction spécialisée}{fonction!spécialisée}
la réduction d'une fonction source obtenue suite à l'application d'une ou
plusieurs actions. Ce chapitre expose comment est représentée
cette réduction, et comment elle est calculée.\\

La réduction d'une fonction doit distinguer les instructions visibles
de celles qui ne le sont pas.
Une information booléenne, attachée à chaque instruction,
devrait donc suffire à
indiquer si celle-ci appartient ou non à la fonction spécialisée, mais
la mise en place d'un outil de navigation incite à enrichir les informations
calculées. On décide donc d'avoir
une annotation plus précise du flot de données, que l'on appelle
{\bf marquage},
pour préciser la raison de la présence ou de l'absence d'un certain élément.
Cela peut permettre de visualiser l'impact des instructions d'une
fonction sur un point de programme (contrôle, données utilisées, déclaration
nécessaires, etc.)\\

Le document qui présente le calcul de PDG expose les éléments qui composent le
graphe de dépendance. La plupart correspondent à une instruction.
Quelques exceptions néanmoins~:
\begin{itemize}
\item les éléments représentant les entrées/sorties d'une fonction
  ne sont pas associés à une instruction,
\item le label d'une instruction est représenté par un élément en plus de ceux
qui représentent l'instruction,
\item un appel de fonction est représenté par plusieurs éléments.\\
\end{itemize}

Nous ne nous considérons pas, dans un premier temps,
la spécialisation des fonctions appelées qui sera étudiée
dans le chapitre \ref{sec-interproc}. 
L'appel de fonction est donc simplement vu pour l'instant comme
une instruction représentée par plusieurs éléments.


Par contre, nous distinguons la visibilité d'un label de celle de l'instruction
associée. Dans la suite, le label sera souvent considéré comme une instruction à
part entière.\\

On peut donc dire que
la fonction spécialisée contient un \indexdef{marquage}
qui fait correspondre une \indexdef{marque} 
aux instructions et labels d'une fonction.

\section{Marquage}

On calcule le marquage d'une fonction en réponse à une requête.
Celle-ci se traduit en général en terme d'éléments du PDG.\\

Initialement, le marquage
peut contenir le résultat de précédents calculs ou être
nouvellement créé, c'est-à-dire vide.\\

Le calcul élémentaire est très simple~:
il consiste à parcourir le PDG,
à calculer la marque pour chaque élément,
et à l'associer à l'instruction ou au label correspondant
en la combinant avec l'éventuelle valeur précédente.\\

Une idée de l'algorithme appliqué est présenté en \ref{sec-algo-mark-fct}.


\subsection{Passage à un point de programme}

Une première requête consiste à marquer les éléments du flot
qui servent à déterminer ce qui permet de passer
à un point de programme.\\

La marque propagée s'appelle
\indextxtdef{marque de contrôle}{marque!de contrôle},
et se note {\it mC}.
% , mais on utilise une
% \indextxtdef{marque de contrôle initiale}{marque!de contrôle initiale}
% ($mC_0$) pour annoter le point de
% départ de la recherche.
% Cela permet de retrouver ce point par la suite, et de ne
% pas sélectionner l'instruction concernée.\\

Pour faire ce calcul, on propage la marque dans les dépendances de contrôle du
point choisi, puis récursivement dans toutes les dépendances des points ainsi
sélectionnés. Cela correspond à marquer  {\it mC} tous les éléments
de l'ensemble $R_{L0}$.

\begin{exemple}
\begin{tabular}{l|p{8cm}}
\begin{tabular}{>{\itshape}c c l}
mC  && \verb!x = y+1;!\\
  && \verb!a = 0;!\\
  && \verb!b = 10;!\\
  && \verb!...!\\
mC  && \verb!if (x > 0) {!\\
  && \verb!...!\\
  && \verb!L: a += b;!\\
  && \verb!...!\\
  && \verb!}!\\
\end{tabular}
&
Il s'agit ici de sélectionner ce qui contrôle le passage au point \verbtt{L}.
Le test \verb!x>0! est donc marqué {\it mC} ainsi que le calcul de \verbtt{x}
dont dépend ce test.
\end{tabular}
\end{exemple}

\subsection{Valeur d'une donnée}

On peut également demander à sélectionner ce qui permet de calculer une donnée
à un point de programme.
On peut par exemple demander à ne garder que ce qui permet de calculer l'une des
sorties d'une fonction.\\

La première étape consiste à trouver l'élément, ou les éléments,
correspondant dans le graphe.
Des éléments particuliers représentent les sorties de la fonction.
Pour des calculs internes,
on peut utiliser l'identification d'une affectation pour parler de la donnée
affectée, ou alors
il faut disposer de l'état utilisé lors de la construction du graphe.
On peut alors retrouver les éléments qui ont participé au calcul en un point
de n'importe quelle
donnée, du moins lorsque le point considéré se trouve dans sa portée.\\

La marque associée au calcul d'une donnée s'appelle
\indextxtdef{marque de valeur}{marque!de valeur}
et se note {\it mV}. Elle sert, comme son nom l'indique,
à annoter les éléments qui participent au calcul
de la valeur de la donnée demandée. \\

Les données qui permettent de calculer l'adresse de la case modifiée (partie
gauche de l'affectation)
sont annotés par une \indextxtdef{marque d'adresse}{marque!d'adresse} ($mA$).
Il s'agit par exemple du calcul de l'indice d'un élément de tableau ou d'un
pointeur.\\

Les éléments
qui permettent d'atteindre le point de programme sont marqués {\it mC}.

\begin{exemple}
\begin{tabular}{m{4cm} c | m{8cm}}
\begin{tabular}{>{\itshape}c c l}
mCA  && \verb!x = y+1;!\\
mV  && \verb!b = 10;!\\
  && \verb!...!\\
mC  && \verb!if (x > 0) {!\\
  && \verb!...!\\
  && \verb!L: t[x] = b;!\\
  && \verb!...!\\
  && \verb!}!\\
\end{tabular}
&&
Il s'agit ici de sélectionner ce qui participe au calcul de l'instruction
située au point \verbtt{L}.
Le test \verb!x>0! est donc marqué {\it mC} ainsi que le calcul de \verbtt{x}
dont dépend ce test. La valeur de \verbtt{b} participe au calcul de la partie
droite et est donc marqué {\it mV}. La partie gauche dépend de
\verbtt{x} qui doit donc avoir la marque {\it mA}. On voit donc que \verbtt{x}
cumule deux informations et est donc marqué {\it mCA}.
\end{tabular}
\end{exemple}


\subsection{Éléments superflus}

On a vu que le marquage est relatif aux instructions,
et que certaines instructions (appels de fonctions) sont représentées
par plusieurs éléments du PDG. Si on ne souhaite pas décomposer les
instructions (ie. spécialiser les appels de fonctions),
il peut arriver qu'une même instruction corresponde à des éléments
visibles et d'autres invisibles. Ces derniers, et leurs dépendances,
sont alors marqués
\indextxtdef{superflus}{marque!superflu} (marque $mS$).
Certaines instructions deviennent alors visibles alors qu'elles ne sont pas
strictement nécessaires au calcul demandé.\\

\begin{exemple}
\begin{center}
\begin{tabular}{m{4cm}p{1cm}m{6cm}}
\begin{verbatim}
int G;
int f (int x, int y) {
  G = x + 1;
  return y + 2;
}
int g (void) {
  int a = 1;
  int b = a+1;
  return f (b, a);
}
\end{verbatim}
&&
Si l'utilisateur demande à sélectionner ce qui permet de calculer la valeur
de retour de \verbtt{g}, on n'aurait en fait besoin que de la valeur de \verbtt{a},
mais comme on ne spécialise pas \verbtt{f}, il faut aussi marquer
l'instruction qui calcule \verbtt{b}  comme superflue.
\end{tabular}
\end{center}
\end{exemple}

\subsection{Marques}

En résumé, les marques possibles sont les suivantes~:
\begin{center}
\begin{tabular}{|>{\itshape}c!{:}l|}
  \hline
  mV & marque de valeur \\
  mC & marque de contrôle \\
  mA & marque d'adresse \\
  mS & marque pour un élément superflu\\
  \hline
\end{tabular}
\end{center}

Les marques $mV,mC,mA$ peuvent se superposer lorsqu'un élément participe au
calcul pour plusieurs raisons. On notera par exemple $mVA$ la marque associée à
un élément qui participe à la valeur et au calcul d'adresse. Par la suite,
on appelle marque $mCAV$ toute marque de cette famille.\\

La marque $mS$ est l'élément neutre de la combinaison.
