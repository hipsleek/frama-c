%% --------------------------------------------------------------------------
%% --- Analysis scripts
%% --------------------------------------------------------------------------

\chapter{Compliance}
\label{user-compliance}

\newcommand{\CWE}[1]{\href{https://cwe.mitre.org/data/definitions/#1.html}{CWE-#1}}

% Macros used by the tables in this chapter
\definecolor{lightgray}{gray}{0.9}
% alternate rowcolors for all long-tables
\let\oldlongtable\longtable
\let\endoldlongtable\endlongtable
\renewenvironment{longtable}{\rowcolors{2}{white}{lightgray}\oldlongtable} {
\endoldlongtable}

\FramaC provides sound analyses for several kinds of code defects.
Given the large amount of covered defects and command-line options which
toggle their reporting, we provide in this chapter information about standards
compliance, coding guidelines and related documents, such as the ISO C standard,
CWEs, CERT C, etc.

This chapter is not exhaustive; in particular, some defects are implicitly
checked without any controlling options; others may be affected by
a combination of several options which is hard to precisely express.

Please contact the \FramaC team if you require a thorough evaluation of the
standards cited here, or of different ones. This reference is provided on a
best-effort basis.

\section{Unsupported C99 and C11 Features}
\label{sec:unsupported-iso-c-features}

The table below lists features of the C99 standard (ISO/IEC 9899:1999) which
are currently unsupported by \FramaC. Please contact the \FramaC team if you
would like them be to supported.

\begin{longtable}{m{0.25\textwidth} m{0.70\textwidth}}
  \caption{Unsupported C99 features}\\
  \cellcolor{white}\textbf{{\em Feature}} & \cellcolor{white}\textbf{{\em Details}} \\
  \hline
  \endhead
  \endfoot
  \label{tab:unsupported-c99-features}%
  Complex/Imaginary types & \verb|_Complex|/\verb|_Imaginary| types are
  not natively supported in OCaml. Lack of user demand led to the absence of
  their implementation in Frama-C. \\
  Nested VLAs & Nested variable-length arrays (VLAs), that is,
  \verb|int t[a][a]| where \verb|a| is not a compile-time constant,
  are currently unsupported.
  VLAs with a single variable dimension are supported. \\
  Floating-point environment & Some prototypes and stubs are available in the
  \texttt{fenv.h} and \texttt{fenv.c} files provided in \FramaC's standard
  library (Section~\ref{sec:libc}), but they are incomplete.
\end{longtable}

The table below lists features of the C11 standard (ISO/IEC 9899:2011) which
are currently unsupported by \FramaC. Please contact the \FramaC team if you
would like them be to supported.

\begin{longtable}{m{0.25\textwidth} m{0.70\textwidth}}
  \caption{Unsupported C11 features}\\
  \cellcolor{white}\textbf{{\em Feature}} & \cellcolor{white}\textbf{{\em Details}} \\
  \hline
  \endhead
  \endfoot
  \label{tab:unsupported-c11-features}%
  \verb|_Alignas| / \verb|_Alignof| & Currently not supported by the \FramaC
  kernel. Also, ACSL does not currently include alignment-related predicates. \\
  \verb|_Atomic| & Currently unsupported (ignored) by the Frama-C kernel. \\
  Unicode & \verb|u/U| character constants, \verb|u8/u/U| string literals, and
  \verb|\u| escape sequences are currently unsupported by the lexer.
  \textbf{Note}: in a few cases, such as ACSL specifications, some \verb|\u|
  escape sequences are allowed (e.g. for ACSL symbols such as $\in$,
  $\mathbb{Z}$, etc). \\
  Complex/Imaginary types & As in C99
  (see Table~\ref{tab:unsupported-c99-features}). Note that C11 made complex
  types optional (they were required in C99). \\
  Nested VLAs & As in C99
  (see Table~\ref{tab:unsupported-c99-features}). Note that C11 made
  variable-length types optional (they were required in C99).
\end{longtable}


\section{Frama-C Options Related to C Undefined Behaviors}

This section lists several \FramaC options affecting (either enabling or
disabling) the detection of {\em unspecified} and {\em undefined} behaviors
listed in Annexes J.1 and J.2 of the C11 standard (ISO/IEC 9899:2011).

Note: the ISO C standard does not provide an identifier for each behavior;
therefore, we use the numbers listed in
{\em SEI CERT C Coding Standard's Back Matter}, sections
{\em CC. Undefined Behavior} and {\em DD. Unspecified Behavior}, whenever
possible. These tables can be found at
\url{https://wiki.sei.cmu.edu/confluence/display/c}.
Note that SEI CERT does not list implementation-defined behaviors; in such
cases, we simply refer to the related section in the ISO C standard.

% ugly hack to force hyphenation inside \texttt, for the very long options
\texttt{\hyphenchar\font=`\- \hyphenpenalty=10000 \exhyphenpenalty=-50}

\begin{longtable}{m{0.42\textwidth} m{0.53\textwidth}}
  \caption{Impact of some \FramaC options on a few unspecified and undefined
    behaviors.}\\
  \cellcolor{white}\textbf{{\em Command-line Option}} & \cellcolor{white}\textbf{{\em Affected behaviors}} \\
  \hline
  \endhead
  \endfoot
  \label{tab:undefined-unspecified-behaviors}%
  \texttt{-eva-initialization-padding-globals} & Controls UnsB related to {\em DD.10} for the initialization status of padding in global variables. \\
  \texttt{-eva-warn-pointer-subtraction} & Toggles warnings related to UB {\em CC.48}. \\
  \texttt{-eva-warn-undefined-pointer-comparison} & Toggles warnings related to pointer comparisons (related to UB {\em CC.53}). \\
  \texttt{-initialized-padding-locals} & Toggles UnsB related to {\em DD.10} for local variables. \\
  \texttt{-unspecified-access} & Enables reporting of some instances of UB {\em CC.35}. \\
  \texttt{-warn-invalid-bool} & Toggles reporting of UB {\em CC.12} when applied to values of type \texttt{\_Bool}. \\
  \texttt{-warn-invalid-pointer} & Toggles reporting of UBs {\em CC.46} and {\em CC.62}. \\
  \texttt{-warn-left-shift-negative} & Toggles reporting of UB {\em CC.52}. \\
  \texttt{-warn-pointer-downcast} & Toggles reporting of UB {\em CC.24}. \\
  \texttt{-warn-right-shift-negative} & Toggles reporting of the IDB mentioned in C11 \S 6.5.7.5. \\
  \texttt{-warn-signed-downcast} & Toggles reporting of UB {\em CC.36} for signed types, when converting from a wider to a narrower type. \\
  \texttt{-warn-signed-overflow} & Toggles reporting of UB {\em CC.36} for operations on signed types (except when converting from a wider to a narrower type). \\
  \texttt{-warn-unsigned-downcast} & Toggles reporting of UB {\em CC.36} for unsigned types, when converting from a wider to a narrower type. \\
  \texttt{-warn-unsigned-overflow} & Toggles reporting of a situation similar to UB {\em CC.36}, for operations on {\em unsigned} types, even though they are allowed by C11. \\
\end{longtable}

% disable the hyphenation ugly hack
\texttt{\hyphenchar\font=-1}

\section{RTE categories and C Undefined Behaviors}

This section presents the correspondence between runtime execution (RTE) alarms
emitted by the Eva plugin and C undefined behaviors.

\begin{longtable}{>{\raggedright}m{0.20\textwidth} >{\raggedright\arraybackslash}m{0.26\textwidth} >{\raggedright\arraybackslash}m{0.46\textwidth}}
  \caption{%
    Correspondence between \FramaC's runtime error categories and C's undefined behaviors.
  }\\
  \cellcolor{white}\textbf{{\em RTE Category}} & \cellcolor{white}\textbf{{\em Related UBs}} & \cellcolor{white}\textbf{{\em Notes}} \\
  \hline
  \endhead
  \endfoot
  \label{tab:rte}%
  bool\_value & {\em CC.12} & All values other than $\{0, 1\}$ are trap representations for the \texttt{\_Bool} type. \\
  dangling\_pointer & {\em CC.9}, {\em CC.10}, {\em CC.177} & \\
  differing\_blocks & {\em CC.48} & \\
  division\_by\_zero & {\em CC.45} & \\
  float\_to\_int & {\em CC.17} & \\
  function\_pointer & {\em CC.26}, {\em CC.41} & \\
  freeable & {\em CC.179} & This is not an RTE category, but an alarm related to an ACSL precondition. \\
  index\_bound & {\em CC.49} & \\
  initialization & {\em CC.11}, {\em CC.21}, {\em CC.180} & \\
  mem\_access & {\em CC.33}, {\em CC.43}, {\em CC.47}, {\em CC.62}, {\em CC.64}, {\em CC.176} (among others) & Alignment issues are currently not reported by Frama-C. \\
  overflow & {\em CC.24}, {\em CC.36}, {\em CC.50} &
  This category comprises the following RTEs:
  \texttt{signed\_overflow}, \texttt{unsigned\_overflow},
  \texttt{signed\_downcast} and \texttt{unsigned\_downcast}
  (mostly related to {\em CC.36}), and
  \texttt{pointer\_downcast} (related to {\em CC.24}).
  For {\em CC.50}, see {\em Note about \texttt{ptrdiff\_t}}. \\
  overlap & {\em CC.54} & {\em CC.100} is handled by \FramaC's libc ACSL specifications. \\
  pointer\_value & {\em CC.46}, {\em CC.62} & \\
  ptr\_comparison & {\em CC.53} & \\
  separation & & Related to ACSL memory separation hypotheses. \\
  shift & {\em CC.51}, {\em CC.52} & \\
  special\_float & & Non-finite floating-point values are not UB, but can be optionally considered as undesirable. \\
\end{longtable}

Note that some undefined behaviors,
such as {\em CC.100}, {\em CC.191}, {\em CC.192} and others,
are not handled by specific categories of RTEs, but instead by ACSL
specifications in \FramaC's libc. These specifications are used by some
analyzers and result in warnings and errors when violated.

\paragraph{Note about \texttt{ptrdiff\_t}}
          {\em CC.50} deals with pointer subtractions, related to type \texttt{ptrdiff\_t}.
          \FramaC does not perform specific handling for this type,
          but in all standard machdeps defined in \FramaC, its definition
          is such that pointer subtraction will lead to a signed overflow
          if the difference cannot be represented in \texttt{ptrdiff\_t},
          thus preventing the undefined behavior. However, if option
          \texttt{-no-warn-signed-overflow} is used, or in a custom machdep,
          this may not hold.

\section{C Undefined Behaviors {\em not} handled by \FramaC}

This section lists some of the C undefined behaviors which are currently
{\em not} directly covered by the open-source version of \FramaC.

The list includes UBs which are delegated to other tools, such as the
preprocessor or the compiler.
\FramaC does not preprocess the sources, relying on external
preprocessors such as GCC's or Clang; therefore, related UBs are out of the
scope of \FramaC and listed below, even though in practice they are verified
by the underlying preprocessor in all but the most exotic architectures.

For a few UBs of syntactic nature, which are always detected during compilation,
\FramaC delegates the task to the compiler. This implies that, when running
\FramaC directly on the code, it may not complain about them; but the code
will generate an error during compilation anyway, {\em without} requiring
specific compilation flags.

This list is not exhaustive; in particular, some UBs not listed here are
partially handled by \FramaC. Please contact the \FramaC team if you would like
more information about whether a specific set of UBs is handled by the platform.

\begin{longtable}{>{\raggedright}m{0.1\textwidth} m{0.9\textwidth} >{\raggedright\arraybackslash}m{0.3\textwidth}}
  \caption{%
    C undefined behaviors {\em not} handled by \FramaC.
  }\\
  \cellcolor{white}\textbf{{\em UB}} & \cellcolor{white}\textbf{{\em Notes}} \\
  \hline
  \endhead
  \endfoot
  \label{tab:ubs-not-handled}%
  CC.2 & Syntactic; out of the scope of \FramaC's analyzers.\\
  CC.3 & Out of scope (lexing/preprocessing-related).\\
  CC.5 & Requires concurrency analysis; the Mthread plugin can provide some guarantees.\\
  CC.14 & Only applies to implementations not following IEEE-754. \FramaC's machdeps assume they do.\\
  CC.18 & Only applies to implementations not following IEEE-754. \FramaC's machdeps assume they do.\\
  CC.27 & Out of scope (lexing/preprocessing-related).\\
  CC.28 & Out of scope (lexing/preprocessing-related).\\
  CC.30 & Out of scope (lexing/preprocessing-related).\\
  CC.31 & Out of scope (lexing/preprocessing-related).\\
  CC.34 & Out of scope (lexing/preprocessing-related).\\
  CC.58 & Syntactic; delegated to the compiler.\\
  CC.59 & Syntactic; delegated to the compiler.\\
  CC.90 & Out of scope (lexing/preprocessing-related).\\
  CC.91 & Out of scope (lexing/preprocessing-related).\\
  CC.92 & Out of scope (lexing/preprocessing-related).\\
  CC.93 & Out of scope (lexing/preprocessing-related).\\
  CC.94 & Out of scope (lexing/preprocessing-related).\\
  CC.95 & Out of scope (lexing/preprocessing-related).\\
  CC.96 & Out of scope (lexing/preprocessing-related).\\
\end{longtable}

\section{Common Weakness Enumerations (CWEs) Reported and not Reported by Frama-C}

This section lists some CWEs handled by \FramaC, as well as some
which are currently out of the scope of \FramaC's default plugins. The latter
may be handled by specialized analyses, such as plugins not currently
distributed with the open-source release of \FramaC, or in future evolutions
of the \FramaC platform.

Note that, due to the large amount of existing CWEs, and the fact that mapping
them to undefined behaviors is not always straightforward, there are likely
several other CWEs which are already handled by \FramaC, at least in some
contexts.
Please contact the \FramaC team if you would like more information about whether
a specific set of CWEs is handled by some analysis in the platform.

Remark: this section is partially based on work related to
NIST's Software Assurance Metrics and Tool Evaluation
(SAMATE\footnote{\url{https://samate.nist.gov/Main_Page.html}}),
namely its Static Analysis Tool Expositions (SATE),
which allowed us to identify some of the CWEs which
\FramaC already can or cannot handle. The CWEs listed here are mostly those
present in Juliet's 1.3 C/C++ test suite, available at NIST SAMATE's website.
Several caveats apply; for instance, some Juliet tests marked as {\em bad}
(that is, exhibiting a weakness) only do so for specific architectural
configurations, e.g. 32-bit pointers. If \FramaC is run with a different
\texttt{machdep}, the test may not exhibit any undefined behavior, as
the weakeness is not actually present for such architectures.
A rigorous assessment of such situations is necessary for strong claims of
soundness.

In Table~\ref{tab:cwe}, the {\em Status} column summarizes the current
handling of the CWE by \FramaC, as one of the following:

\begin{description}
\item[Handled] means \FramaC already handles the CWE; in some cases, its
  detection may be controlled by command-line options;
\item[Partially Handled] means the CWE is handled only in some specific cases;
  \FramaC can thus help {\em detect} occurrences of this CWE, but not
  {\em prove their absence};
\item[Annotations] means the CWE could be handled by \FramaC, but some
  development is required; in practice, this often means user annotations
  will be required, e.g. for specifying the source of tainted data;
  this often implies adding a new abstract domain to Eva, or modifying the
  analyzer to propagate the data provided by the annotations;
\item[Syntactic] means the CWE is of a syntactic nature, while \FramaC
  specializes in semantic analyses; it is currently not handled by \FramaC
  and could probably be so, if a user required it; but it is not the primary
  concern of the platform developers.
\item[Too Vague] means the CWE is either too vague or too broad to be formally
  matched to a given set of properties that \FramaC can verify.
\end{description}

{\small
\begin{longtable}{>{\raggedright}m{0.42\textwidth} m{0.13\textwidth} >{\raggedright\arraybackslash}m{0.45\textwidth}}
  \caption{%
    CWEs handled and not currently handled by \FramaC's open-source plugins.
    The {\em Notes} column may contain references to the command-line options
    table (\ref{tab:undefined-unspecified-behaviors}).
    The descriptions in the {\em Status} column are detailed above.
  }\\
  \cellcolor{white}\textbf{{\em CWE}} & \cellcolor{white}\textbf{{\em Status}} & \cellcolor{white}\textbf{{\em Notes}} \\
  \hline
  \endhead
  \endfoot
  \label{tab:cwe}%
  \CWE{15}: External Control of System or Configuration Setting & Annotations & Requires annotating configuration settings and untrusted sources\\
  \CWE{23}: Relative Path Traversal & Annotations & Requires annotating path-related functions and untrusted sources\\
  \CWE{36}: Absolute Path Traversal & Annotations & Requires annotating path-related functions and untrusted sources\\
  \CWE{78}: Improper Neutralization of Special Elements used in an OS Command ('OS Command Injection') & Annotations & Requires annotating which functions may inject OS commands\\
  \CWE{90}: Improper Neutralization of Special Elements used in an LDAP Query ('LDAP Injection') & Annotations & Requires annotating which functions are LDAP-related\\
  \CWE{114}: Process Control & Annotations & Requires annotating sensitive functions and untrusted sources\\
  \CWE{119}: Improper Restriction of Operations within the Bounds of a Memory Buffer & Handled & -\\
  \CWE{120}: Buffer Copy without Checking Size of Input ('Classic Buffer Overflow') & Handled & -\\
  \CWE{121}: Stack-based Buffer Overflow & Handled & -\\
  \CWE{122}: Heap-based Buffer Overflow & Handled & -\\
  \CWE{123}: Write-what-where Condition & Handled & -\\
  \CWE{124}: Buffer Underwrite & Handled & -\\
  \CWE{125}: Out-of-bounds Read & Handled & -\\
  \CWE{126}: Buffer Overread & Handled & -\\
  \CWE{127}: Buffer Underread & Handled & -\\
  \CWE{128}: Wrap-around Error & Handled & Toggled by options \texttt{-warn-signed-overflow}, \texttt{-warn-unsigned-overflow}, \texttt{-warn-signed-downcast} and \texttt{-warn-unsigned-downcast}.\\
  \CWE{129}: Improper Validation of Array Index & Handled & -\\
  \CWE{131}: Incorrect Calculation of Buffer Size & Handled & Reported not directly at the allocation site, but during usage; the GUI allows navigating back to the source.\\
  \CWE{134}: Use of Externally-Controlled Format String & Annotations & Requires annotating which format strings come from external sources\\
  \CWE{176}: Improper Handling of Unicode Encoding & Annotations & Requires annotating Unicode-related functions and variables\\
  \CWE{188}: Reliance on Data Memory Layout & Partially Handled & \FramaC memory model handles some kinds of invalid accesses\\
  \CWE{190}: Integer Overflow or Wraparound & Handled & See remarks about {\em CC.17} in Table~\ref{tab:undefined-unspecified-behaviors}\\
  \CWE{191}: Integer Underflow & Handled & See remarks about {\em CC.17} in Table~\ref{tab:undefined-unspecified-behaviors}\\
  \CWE{194}: Unexpected Sign Extension & Handled & See note about {\em Numerical Conversions}\\
  \CWE{195}: Signed to Unsigned Conversion Error & Handled & See note about {\em Numerical Conversions}\\
  \CWE{196}: Unsigned to Signed Conversion Error & Handled & See note about {\em Numerical Conversions}\\
  \CWE{197}: Numeric Truncation Error & Handled & See note about {\em Numerical Conversions}\\
  \CWE{222}: Truncation of Security Relevant Information & Not Handled & -\\
  \CWE{223}: Omission of Security Relevant Information & Not Handled & -\\
  \CWE{226}: Sensitive Information in Resource Not Removed Before Reuse & Annotations & Requires annotating shared resources and external entities\\
  \CWE{242}: Use of Inherently Dangerous Function & Annotations & Requires annotating which functions are ``inherently dangerous''\\
  \CWE{244}: Improper Clearing of Heap Memory Before Release ('Heap Inspection') & Not Handled & Semantic property unavailable at the C memory model; syntactic heuristics can be devised\\
  \CWE{252}: Unchecked Return Value & Syntactic & -\\
  \CWE{253}: Incorrect Check of Function Return Value & Syntactic + Annotations & Similar to \CWE{252}, but also requires annotations definining what ``correct check'' means\\
  \CWE{256}: Unprotected Storage of Credentials & Annotations & Requires annotating credential-related functions and variables\\
  \CWE{259}: Use of Hard-coded Password & Annotations & Requires annotating password-related functions and variables\\
  \CWE{272}: Least Privilege Violation & Annotations & Requires annotating sensitive functions\\
  \CWE{273}: Improper Check for Dropped Privileges & Annotations & Requires annotating sensitive functions and forbidden control paths\\
  \CWE{284}: Improper Access Control & Annotations & Requires annotating privileges, permissions, control paths, etc.\\
  \CWE{319}: Cleartext Transmission of Sensitive Information & Annotations & Requires annotating sensitive data and transmission-related functions\\
  \CWE{321}: Hard Coded Cryptographic Key & Annotations & Requires annotating which data correspond to cryptographic keys\\
  \CWE{325}: Missing Cryptographic Step & Annotations & Requires annotating sequences of valid cryptographic function calls\\
  \CWE{327}: Use of a Broken or Risky Cryptographic Algorithm & Annotations & Requires annotating which algorithms are ``broken or risky''\\
  \CWE{328}: Reversible One-Way Hash & Annotations & Requires annotating hash-related functions and variables\\
  \CWE{338}: Use of Cryptographically Weak Pseudo-Random Number Generator (PRNG) & Annotations & Requires annotating PRNG-related functions and variables\\
  \CWE{364}: Signal Handler Race Condition & Not Handled & -\\
  \CWE{366}: Race Condition Within Thread & Not Handled & Some situations can be handled by the Mthread plugin\\
  \CWE{367}: TOC TOU & Not Handled & -\\
  \CWE{369}: Divide by Zero & Handled & -\\
  \CWE{377}: Insecure Temporary File & Annotations & Requires annotating functions and data flows related to temporary files\\
  \CWE{390}: Error Without Action & Not Handled & -\\
  \CWE{391}: Unchecked Error Condition & Not Handled & -\\
  \CWE{400}: Uncontrolled Resource Consumption & Annotations & Requires annotating resource-related functions and variables\\
  \CWE{401}: Missing Release of Memory after Effective Lifetime & Not Handled & -\\
  \CWE{404}: Improper Resource Shutdown or Release & Annotations & Requires annotating resources and functions manipulating them\\
  \CWE{415}: Double Free & Handled & -\\
  \CWE{416}: Use After Free & Handled & -\\
  \CWE{426}: Untrusted Search Path & Not Handled & Not UB-related; requires annotations\\
  \CWE{427}: Uncontrolled Search Path Element & Annotations & Requires annotating path-related functions and untrusted sources\\
  \CWE{440}: Expected Behavior Violation & Too Vague & -\\
  \CWE{457}: Use of Uninitialized Variable & Handled & See remarks about {\em DD.10}\\
  \CWE{459}: Incomplete Cleanup & Annotations & Requires annotating resources and cleanup functions\\
  \CWE{464}: Addition of Data Structure Sentinel & Not Handled & -\\
  \CWE{467}: Use of sizeof on a Pointer Type & Syntactic & -\\
  \CWE{468}: Incorrect Pointer Scaling & Syntactic & -\\
  \CWE{469}: Use of Pointer Subtraction to Determine Size & Handled & -\\
  \CWE{475}: Undefined Behavior for Input to API & Too Vague & \FramaC already handles some cases related to the C standard library, but in general, may require annotations\\
  \CWE{476}: NULL Pointer Dereference & Handled & -\\
  \CWE{478}: Missing Default Case in Switch & Syntactic & -\\
  \CWE{479}: Signal Handler Use of Non Reentrant Function & Not Handled & -\\
  \CWE{480}: Use of Incorrect Operator & Too Vague & -\\
  \CWE{481}: Assigning Instead of Comparing & Syntactic & -\\
  \CWE{482}: Comparing Instead of Assigning & Syntactic & -\\
  \CWE{483}: Incorrect Block Delimitation & Syntactic & -\\
  \CWE{484}: Omitted Break Statement in Switch & Syntactic & -\\
  \CWE{506}: Embedded Malicious Code & Too Vague & Defining ``malicious code'' automatically is hard; manual annotations defeat the purpose\\
  \CWE{510}: Trapdoor & Too Vague & Sound analyses such as those proposed by \FramaC are able to exhaustively identify some kinds of trapdoors\\
  \CWE{511}: Logic/Time Bomb & Not Handled & -\\
  \CWE{526}: Exposure of Sensitive Information Through Environmental Variables & Annotations & Requires annotating sensitive information and environment-related functions and variables\\
  \CWE{546}: Suspicious Comment & Syntactic & -\\
  \CWE{561}: Dead Code & Partially Handled & Metrics and Eva provide information about syntactic and semantic coverage\\
  \CWE{562}: Return of Stack Variable Address & Handled & Related to Eva's warning category \texttt{locals-escaping}\\
  \CWE{563}: Unused Variable & Syntactic & Mostly syntactic in nature; compilers often warn about it\\
  \CWE{570}: Expression Always False & Syntactic & Mostly syntactic in nature; compilers often warn about it\\
  \CWE{571}: Expression Always True & Syntactic & Mostly syntactic in nature; compilers often warn about it\\
  \CWE{587}: Assignment of Fixed Address to Pointer & Handled & Detected via \texttt{-warn-invalid-pointer} at the assignment, otherwise indirectly at the point of usage; option \texttt{-absolute-valid-range} changes its behavior\\
  \CWE{588}: Attempt to Access Child of a Non-structure Pointer & Partially Handled & \FramaC emits warnings for certain types of incompatible casts\\
  \CWE{590}: Free Memory Not on Heap & Handled & -\\
  \CWE{591}: Sensitive Data Storage in Improperly Locked Memory & Not Handled & -\\
  \CWE{605}: Multiple Binds to the Same Port & Annotations & Requires annotating socket-related functions and variables\\
  \CWE{606}: Unchecked Loop Condition & Annotations & Requires annotating which data is user-supplied\\
  \CWE{615}: Inclusion of Sensitive Information in Source Code Comments & Not Handled & -\\
  \CWE{617}: Reachable Assertion & Handled & \FramaC's libc \texttt{assert} specification contains an ACSL assertion\\
  \CWE{620}: Unverified Password Change & Not Handled & -\\
  \CWE{665}: Improper Initialization & Handled & -\\
  \CWE{666}: Operation on Resource in Wrong Phase of Lifetime & Annotations & Requires annotating resources and their lifecycles\\
  \CWE{667}: Improper Locking & Annotations & Requires annotating locks and functions manipulating them\\
  \CWE{672}: Operation on Resource After Expiration or Release & Annotations & Requires annotating resources and their lifecycles\\
  \CWE{674}: Uncontrolled Recursion & Too Vague & The identification of an ``uncontrolled'' recursion is complex\\
  \CWE{675}: Duplicate Operations on Resource & Annotations & Requires annotating resources and operations on them\\
  \CWE{676}: Use of Potentially Dangerous Function & Annotations & Requires annotating which functions are ``potentially dangerous''\\
  \CWE{680}: Integer Overflow to Buffer Overflow & Handled & -\\
  \CWE{681}: Incorrect Conversion Between Numeric Types & Partially Handled & See note about {\em Numerical Conversions}\\
  \CWE{685}: Function Call With Incorrect Number of Arguments & Partially Handled & The Variadic plugin handles most cases related to variadic function calls\\
  \CWE{688}: Function Call With Incorrect Variable or Reference as Argument & Partially Handled & Some cases are related to variadic functions (e.g. \texttt{printf}) and detected by the Variadic plugin\\
  \CWE{690}: Unchecked Return Value to NULL Pointer Dereference & Handled & For functions related to dynamically allocated memory, toggled via option \texttt{-eva-alloc-returns-null}\\
  \CWE{758}: Undefined Behavior & Partially Handled & The C language has too many undefined behaviors, but Frama-C does handle several of them\\
  \CWE{761}: Free Pointer Not at Start of Buffer & Handled & -\\
  \CWE{762}: Mismatched Memory Management Routines & Annotations & Requires annotating memory management functions and objects\\
  \CWE{773}: Missing Reference to Active File Descriptor or Handle & Annotations & Requires annotating resources and operations on them\\
  \CWE{775}: Missing Release of File Descriptor or Handle & Annotations & Requires annotating resources and their lifecycles\\
  \CWE{780}: Use of RSA Algorithm Without OAEP & Not Handled & -\\
  \CWE{785}: Path Manipulation Function Without Max Sized Buffer & Annotations & Requires all relevant filepath-related functions to have correct annotations\\
  \CWE{786}: Access of Memory Location Before Start of Buffer & Handled & -\\
  \CWE{787}: Out-of-bounds Write & Handled & -\\
  \CWE{788}: Access of Memory Location After End of Buffer & Handled & -\\
  \CWE{789}: Memory Allocation with Excessive Size Value & Annotations & Requires annotating memory limits\\
  \CWE{823}: Use of Out-of-range Pointer Offset & Handled & -\\
  \CWE{824}: Access of Uninitialized Pointer & Handled & -\\
  \CWE{825}: Expired Pointer Dereference & Handled & -\\
  \CWE{832}: Unlock of Resource That is Not Locked & Annotations & Requires annotating resources and operations on them\\
  \CWE{835}: Infinite Loop & Partially Handled & Loops which, semantically, are {\em always} infinite are reported by the Nonterm plugin\\
  \CWE{843}: Access of Resource Using Incompatible Type ('Type Confusion') & Not Handled & -\\
  \CWE{912}: Hidden Functionality & Too Vague & \FramaC's sound analysis can show the absence of backdoors, but only if they can be semantically specified (e.g. via annotations)\\
\end{longtable}
}

\paragraph{Note about {\em Numerical Conversions}}

A few CWEs (\CWE{194}, \CWE{195}, \CWE{196} and \CWE{197}) related to numerical
conversion issues do not map directly to undefined behaviors
(except, in some cases, to {\em CC.36});
some of them are related to unspecified and implementation-defined behaviors.
\FramaC has a set of command-line options to enable warnings related to these
conversions, when they can overflow or generate unexpected values:
\texttt{-warn-signed-overflow}, \texttt{-warn-unsigned-overflow},
\texttt{-warn-signed-downcast}, and \texttt{-warn-unsigned-downcast}.

See Section~\ref{sec:customizing-analyzers} for more details and some examples
related to these options.

Also note that conversion from a floating-point value to an integer may
overflow; in this case, an alarm \texttt{float\_to\_int} is generated,
independently of the previous options.

% Undoing the macros used by the tables in this chapter
% restore longtables
\renewenvironment{longtable}{\oldlongtable} {
\endoldlongtable}
