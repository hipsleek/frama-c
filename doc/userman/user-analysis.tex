\chapter{Platform-wide Analysis Options}
\label{user-analysis}

The options described in this chapter provide each analysis with common
hypotheses that influence directly their behavior.  For this reason, the user
must understand them and the interpretation the relevant plug-ins have of them.
Please refer to individual plug-in documentations
(e.g.~\cite{value,slicing,wp}) for specific options.

%They were first
%specific to the Value Analysis plug-in. As they are not so specific, they have
%been translated from this plug-in to the \FramaC kernel in the Beryllium
%release of \FramaC. However, at this day, they are not supported by all
%plug-ins and users are invited to examine their precise effects in the
%corresponding manuals.

\section{Entry Point}

The following options define the entry point of the program and related initial
conditions.
\begin{description}
\item \texttt{\optiondef{-}{main} <function\_name>} specifies that all
  analyzers should treat function \texttt{function\_name} as the entry
  point of the program.
\item \optiondef{-}{lib-entry} indicates that analyzers should not assume
  globals to have their initial values at the beginning of the analysis. This
  option, together with the specification of an entry point \texttt{f}, can be
  used to analyze the function \texttt{f} outside of a calling context, even if
  it is not the actual entry point of the analyzed code.
\end{description}

\section{Feedback Options}\label{sec:feedback-options}

All \FramaC plug-ins define the following set of common options.
\begin{description}
\item[\texttt{-<plug-in shortname>-help}] (or \texttt{-<plug-in shortname>-h})
  \optionidxdef{-}{<plug-in>-help}
  \optionidxdef{-}{kernel-help}
  prints out the list of options of the given plug-in.

\item[\texttt{-<plug-in shortname>-verbose <n>}]
  \optionidxdef{-}{<plug-in>-verbose}
  \optionidxdef{-}{kernel-verbose}
 sets the level of verbosity to
  some positive integer \texttt{n}. A value of 0 means no information
  messages. Default is 1.

\item[\texttt{-<plug-in shortname>-debug <n>}]
  \optionidxdef{-}{<plug-in>-debug}
  \optionidxdef{-}{kernel-debug}
  sets the debug level to a
  positive integer \texttt{n}. The higher this number, the more debug messages
  are printed. Debug messages do not have to be understandable by the end
  user. This option's default is 0 (no debugging messages).
\item[\texttt{-<plug-in shortname>-msg-key <keys>}]
  \optionidxdef{-}{<plug-in>-msg-key}
  \optionidxdef{-}{kernel-msg-key}
  sets the categories of
  messages that must be output for the plugin. \texttt{keys} is a
  comma-separated list of names. The list of available categories can be
  obtained with \texttt{-<plug-in shortname>-msg-key help}. To enable all
  categories, use the wildcard \texttt{'*'}%
  \footnote{Be sure to enclose it in single quotes or your shell might
    expand it, leading to unexpected results.}. Categories can have
  subcategories, defined by a colon in their names. For instance, \texttt{a:b:c}
  is a subcategory \texttt{c} of \texttt{a:b}, itself a
  subcategory of \texttt{a}. Enabling a category will also enable all its
  subcategories. An enabled category \texttt{cat} can be disabled by using
  \texttt{-cat} in the list of \texttt{keys}. Several occurrences of the option
  may appear on the command line and will be processed in order.
\item[\texttt{-<plug-in shortname>-warn-key <keys>}]
\optionidxdef{-}{<plug-in>-warn-key}
\optionidxdef{-}{kernel-warn-key}
allows setting the status of a category of warnings. The argument \texttt{keys} is a comma-separated
list of \texttt{key} of the form \texttt{<category>=<status>}, where category is a warning
category (possibly a sub-category as for messages categories above), and status is one of:
\index{warning status}
\begin{description}
\item[inactive] no message is emitted for the category
\item[feedback] a feedback message is emitted
\item[active] a proper warning is emitted
\item[once] a proper warning is emitted, and the status of the category is
  reset to \texttt{inactive}, {\it i.e.}
  at most one message for the category will be emitted.
\item[error] a warning is emitted. \FramaC execution continues, but its
  exit status will not be 0 at the end of the run.
\item[abort] a warning is emitted and \FramaC will immediately abort
 its execution with an error.
\item[feedback-once] a feedback message is emitted, and the status is reset to
  \texttt{inactive}
\item[err-once] combines the actions of \texttt{error} and \texttt{once}
  statuses.
\end{description}

The \texttt{=<status>} part might be omitted, which is equivalent to asking for \texttt{active} status.
The new status will be propagated to subcategories, with one exception: statuses
``abort'', ``error'' and ``err-once'' will only be propagated to subcategories
whose current status is not ``inactive''.

Finally, as for debug categories, passing \texttt{help} (without status) in the list of
\texttt{keys} will list the available categories, together with their
current status. Passing \texttt{*} in the list of keys will change the status
of all warning categories, and affect warnings that do not have a category.
Hence, \texttt{-kernel-warn-key *=abort} will stop \FramaC's execution at the
first warning triggered by the kernel.
\end{description}

The two following options modify the behavior of output messages:
\begin{description}
\item[\optiondef{-}{add-symbolic-path}] takes a list of the form $path_1:name_1,
  \dots, path_n:name_n$ in argument and replaces each $path_i$ with $name_i$ when
  displaying file locations in messages.
\item[\optiondef{-}{permissive}] performs less verification on validity of
  command-line options.
\end{description}

\section{Customizing Analyzers}\label{sec:customizing-analyzers}

The descriptions of the analysis options follow. For the first two, the
description comes from the \Value manual~\cite{value}. Note that
these options are very likely to be modified in future versions of \FramaC.

\begin{description}
\item \optiondef{-}{absolute-valid-range} \texttt{m-M}
specifies that the only valid absolute addresses (for reading or writing)
are those comprised between \lstinline$m$ and \lstinline$M$ inclusive.
This option currently allows to specify only a single interval,
although it could be improved to allow several intervals
in a future version. $m$ and $M$
can be written either in decimal or hexadecimal notation.

\item \optiondef{-}{unsafe-arrays} can be used when the source
code manipulates n-dimensional arrays, or arrays within structures,
in a non-standard way. With this option, accessing
indexes that are out of bounds will instead access
the remainder of the struct. For example, the code below will overwrite
the fields \lstinline|a| and \lstinline|c| of \lstinline|v|.
\begin{ccode}
struct s {
  int a;
  int b[2];
  int c;
};

void main(struct s v) {
  v.b[-1] = 1;
  v.b[2] = 4;
}
\end{ccode}
The opposite option, called \optiondef{-}{safe-arrays}, is set by default.
With \texttt{-safe-arrays}, the two accesses to \lstinline|v| are considered
invalid. (Accessing \lstinline|v.b[-2]| or \lstinline|v.b[3]| remains incorrect,
regardless of the value of the option.)

\item \optiondef{-}{warn-invalid-pointer} may be used to check that the code
  does not perform illegal pointer arithmetics, creating pointers that do not
  point inside an object or one past an object.
  This option is disabled by default, allowing the creation of such invalid
  pointers without alarm — but the dereferencing of an invalid pointer
  \emph{always} generates an alarm.

  For instance, no error is detected by default in the following example, as
  the dereferencing is correct. However, if option
  \texttt{-warn-invalid-pointer} is enabled, an error is detected at line 4.
  \begin{ccode}
    int x;
    int *p = &x;
    p++;  // valid
    p++;  // undefined behavior
    *(p-2) = 1;
  \end{ccode}

  \begin{important}
    Currently, the option is disabled by default. The rationale for this is the
    fact that it creates a lot of redundancies with pointer access alarms and
    that most of the time even when a pointer reaches such an invalid value, it
    is never read again, or if it is, the memory access will trigger an alarm.
  \end{important}

\item \optiondef{-}{unspecified-access} may be used to check when the
  evaluation of an expression depends on the order in which its sub-expressions
  are evaluated. For instance, this occurs with the following piece of code.
\begin{ccode}
int i, j, *p;
i = 1;
p = &i;
j = i++ + (*p)++;
\end{ccode}
In this code, it is unclear in which order the elements of the right-hand side
of the last assignment are evaluated.  Indeed, the variable j can get any value
as i and p are aliased. The \texttt{-unspecified-access} option warns
against such ambiguous situations. More precisely, \texttt{-unspecified-access}
detects potential concurrent write accesses (or a write access
and a read access) over the same location that are not separated by a sequence
point. Note however that this option \emph{does not warn} against such accesses
if they occur in an inner function call, such as in the following example:
\begin{ccode}
int x;
int f() { return x++; }
int g() { return f() + x++; }
\end{ccode}
Here, the \texttt{x} might be incremented by \texttt{g} before or after the
call to \texttt{f}, but since the two write accesses occur in different
functions, \texttt{-unspecified-access} does not detect that.

\item \optiondef{-}{warn-pointer-downcast} may be used to check that the code
  does not downcast a pointer to an integer type. This option is set by default.
  In the following example, analyzers report by default an error on the third
  line. Disabling the option removes this verification.
  \begin{ccode}
    int x;
    uintptr_t addr = &x;
    int a = &x;
  \end{ccode}

\item \optiondef{-}{warn-signed-downcast} may be used to check that the analyzed
  code does not downcast an integer to a signed integer type. This option is
  \emph{not} set by default. Without it, the analyzers do not perform such a
  verification. For instance consider the following function.
\begin{ccode}
short truncate(int n) {
  return (short) n;
}
\end{ccode}
If \texttt{-warn-signed-downcast} is set, analyzers report an error on
{\lstset{language=C,style=frama-c-style} \lstinline|(short) n|}
which downcasts a signed integer to a signed
short. Without it, no error is reported.

\item \optiondef{-}{warn-unsigned-downcast} is the same as
  \texttt{-warn-signed-downcast} for downcasts to unsigned integers. This option
  is also \emph{not} set by default.

\item \optiondef{-}{warn-signed-overflow} may be used to check that the
  analyzed code does not overflow on integer operations. If the opposite option
  \texttt{-no-warn-signed-overflow} is specified, the analyzers assume that
  operations over signed integers may overflow by following two's complement
  representation. This option is set by default. For instance, consider the
  function \lstinline|abs| that computes the absolute value of its
  \lstinline|int| argument.
\begin{ccode}
int abs(int x) {
  if (x < 0) x = -x;
  return x;
}
\end{ccode}
By default, analyzers detect an error on
\lstinline|-x| since this operation overflows when \lstinline|MININT| is the
argument of the function. But, with the \texttt{-no-warn-signed-overflow}
option, no error is detected.

\item \optiondef{-}{warn-unsigned-overflow} is the same as
  \texttt{-warn-signed-overflow} for operations over unsigned integers. This
  option is \emph{not} set by default.

\item \optiondef{-}{warn-left-shift-negative} can be used to check that the
  code does not perform signed left shifts on negative values, i.e.,
  \lstinline|x << n| with \lstinline|x| having signed type and negative
  value. This is set by default, and can be disabled with option
  \texttt{-no-warn-left-shift-negative}.

\item \optiondef{-}{warn-right-shift-negative}, as its left-shift counterpart,
  can be used to check for negative {\em right} shifts, i.e.,
  \lstinline|x >> n| with \lstinline|x| having signed type and negative value.
  This is {\em not} set by default.
  \texttt{-no-warn-right-shift-negative} can be used to disable the option if
  previously enabled.

\item \optiondef{-}{warn-special-float} \texttt{<type>} may be used to allow
  or forbid special floating-point values, generating alarms when they are
  produced. \texttt{<type>} can be one of the following values:
  \begin{description}
  \item[\texttt{non-finite}]: warn on infinite floats or NaN
  \item[\texttt{nan}]: warn on NaN only
  \item[\texttt{none}]: no warnings
  \end{description}

\item \optiondef{-}{warn-invalid-bool} may be used to check that the code
  does not use invalid \_Bool values by reading trap representations
  from lvalues of \_Bool types. A trap representation does not represent
  a valid value of the \_Bool type, which can only be 0 or 1.
  This option is set by default, and can be disabled with
  \texttt{-no-warn-invalid-bool}.
\end{description}

% Local Variables:
% TeX-master: "userman.tex"
% ispell-local-dictionary: "english"
% compile-command: "make"
% End:
