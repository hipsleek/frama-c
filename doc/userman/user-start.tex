\chapter{Getting Started}
\label{user-start}

This chapter describes \emph{how} to install \FramaC and
\emph{what} this installation provides.

\section{Installation}\label{sec:install}\index{Installation}

The \FramaC platform is distributed as source code, including the \FramaC kernel
and a base set of open-source plug-ins.

The recommended way to install \FramaC is by using the
\opam\index{opam}\footnote{\url{http://opam.ocaml.org}} package manager to
install the \texttt{frama-c} package, which should always point to the latest
release compatible with the configured OCaml compiler. \opam is able to handle
the dependencies required by the \FramaC kernel.

\FramaC can also be installed via pre-compiled binaries,
which include many of the required libraries and other dependencies, although
there is a delay between each new \FramaC release and the availability of a
binary package for the considered platform.

Finally, \FramaC can be compiled and installed from the source distribution,
as long as its dependencies have already been installed. The exact set of
dependencies varies from release to release. They are listed as constraints
in the \texttt{opam} file of the source distribution.

A {\em reference configuration}, guaranteed to be a working set of dependencies
for \FramaC kernel and the open-source plug-ins included in the source
distribution, is available in the \texttt{reference-configuration.md} file of
the source distribution.

For more installation instructions, consider reading the \texttt{INSTALL.md}
file of the source distribution. The main components necessary for compiling
and running \FramaC are described below.

\begin{description}
\item[A \C preprocessor]\index{C preprocessor} is required for \emph{using}
  \FramaC on \C files. If you do not have any \C preprocessor,
  you can only run \FramaC on already preprocessed \texttt{.i} files.
\item[A \C compiler]\index{C compiler} is required to compile the \FramaC
  kernel.
\item[A \tool{Unix}-like compilation environment] with at least the tool
  \texttt{GNU make}\footnote{\url{http://www.gnu.org/software/make}} $\ge 4.0$,
  as well as various POSIX commands, libraries and header files, is necessary
  for compiling \FramaC and its plug-ins.
\item[The \caml compiler]\index{OCaml compiler}\footnote{\url{http://ocaml.org}}
  is required both for compiling \FramaC from source \emph{and} for compiling
  additional plug-ins. Compatible OCaml versions are listed as constraints in
  the \texttt{opam} file.
\end{description}

Other components, such as OcamlGraph, Zarith, Gtk-related packages for the GUI,
etc., are listed in the \texttt{INSTALL.md} and \texttt{opam} files.

\section{One Framework, Several Executables}\label{sec:modes}

The main executables installed by \FramaC are:

\begin{itemize}
\item \texttt{frama-c}\codeidx{frama-c}: batch version (command line);
\item \texttt{frama-c-gui}\codeidx{frama-c-gui}: interactive version (graphical interface).
\end{itemize}

\begin{description}
\item[Batch version]\index{Batch version}

  The \texttt{frama-c} binary can be used to perform most \FramaC analyses:
  from parsing the sources to running complex analyses, on-demand or as part
  of a continuous integration pipeline. Many analyses can display their output
  in simple text form, but some structured results (HTML, CSV or JSON) are
  also available. For instance, a SARIF~(see Section~\ref{sarif}) output
  based on JSON is can be produced by the \tool{Markdown Report} plug-in.
  While the batch version is very powerful, it does not offer the visualization
  features of the interactive version.

\item[Interactive version]\index{Interactive version}
  The interactive version is a GUI that can be used to select the set of files
  to analyze, specify options, launch analyses, browse the code and observe
  analysis results at one's convenience (see Chapter~\ref{user-gui} for
  details). For instance, you can hover over an expression in the code and
  obtain immediate syntactic and semantic information about it; or use context
  menus for easy code navigation.
  However, the initial analysis setup (especially parsing) can be complex for
  some projects, and the batch version is better suited for providing error
  messages at this stage.

\end{description}

Both versions are complementary: the batch version is recommended
for initial setup and analysis, while the interactive version is recommended for
results visualization and interactive proofs.

Besides these two executables, \FramaC also provides a few other command line
binaries:
\begin{itemize}
\item \texttt{frama-c-config}\codeidx{frama-c-config}: auxiliary batch version for
  quickly retrieving configuration information (e.g. installation paths);
  it is only useful for scripting and running a large amount of analyses;
\item \texttt{frama-c-script}\codeidx{frama-c-script}: contains several
  utilities related to source preparation, results visualization and analysis
  automation. Run it without arguments to obtain more details.
\end{itemize}

Finally, note that the two main binaries (\texttt{frama-c} and
\texttt{frama-c-gui} are also provided in {\em bytecode} version, as
\texttt{frama-c.byte} and \texttt{frama-c-gui.byte}.
\index{Native-compiled|bfit}\index{Bytecode|bfit}
The bytecode is sometimes able to provide better debugging information;
but since the native-compiled version is usually much faster (10x),
use the latter unless you have a specific reason to use the bytecode one.

\section{\FramaC Command Line and General Options}\index{Options}

\subsection{Getting Help}

The option \optiondef{-}{help} or \optiondef{-}{h} or
\optiondef{-{}-}{help} gives a very short summary of \FramaC's command line,
with its version and the main help entry points presented below.

The other options of the \FramaC kernel, \ie those which are not specific to any
plug-in, can be printed out through either the option
\optiondef{-}{kernel-help} or \optiondef{-}{kernel-h}.

The list of all installed plug-ins can be obtained via \optiondef{-}{plugins},
while \optiondef{-}{version} prints the \FramaC version only
(useful for installation scripts).

The options of the installed plug-ins are displayed by using either the
option \texttt{-<plug-in shortname>-help} or \texttt{-<plug-in shortname>-h}.

Finally, the option \optiondef{-}{explain} can be used to obtain information
about some specific options of the kernel or of any installed plug-ins:
it prints a help message for each other option given on the command line.

\subsection{\FramaC Configuration}\label{sec:version}

The complete configuration of \FramaC can be obtained with various options,
all documented with \texttt{-kernel-h}:

\begin{quote}
\begin{tabular}{ll}
  \optiondef{-}{print-version} & version number only \\
  \optiondef{-}{print-share-path} & directory for shared resources \\
  \optiondef{-}{print-lib-path} & directory for the \FramaC kernel \\
  \optiondef{-}{print-plugin-path} & directory for installed plug-ins \\
  \optiondef{-}{print-config} & summary of the \FramaC configuration \\
  \optiondef{-}{plugins} & list of installed plug-ins
\end{tabular}
\end{quote}

There are many aliases for these options, but for backward compatibility purposes only.
Those listed above should be used for scripting.
Note that, for all of these options except \texttt{-plugins},
you can use \texttt{frama-c-config} (instead of \texttt{frama-c}),
which offers faster loading times.
This is unnecessary for occasional usage, but when running hundreds of
instances of short analyses on \FramaC, the difference is significant.

For a more thorough display of configuration-related data, in JSON format,
use option \optiondef{-}{print-config-json}. Note that the data output by this
option is likely to change in future releases.

\subsection{Options Outline}

The batch and interactive versions of \FramaC obey a number of
command-line options. Any option that exists in these two modes
has the same meaning in both.
For instance, the batch version can be made to launch the value analysis
on the \verb|foo.c| file with the command \verb|frama-c -eva foo.c|.
Although the GUI allows to select files and to
launch the value analysis interactively, the
command \verb|frama-c-gui -eva foo.c| can be used to launch the value
analysis on the file \verb|foo.c| and immediately start displaying the results
in the GUI.

Any option requiring an argument may use the following
format:
\begin{commands}
\texttt{-option\_name value}
\end{commands}

\paragraph{Parameterless Options}

Most parameterless options have an opposite option, often written by prefixing
the option name with \texttt{no-}.  For instance, the option \verb|-unicode|
for using the Unicode character set in messages has an opposite option
\optionidx{-}{no-unicode} for limiting the messages to ASCII.  Plug-in options
with
a name of the form \texttt{-<plug-in name>-<option name>} have their opposite
option named \texttt{-<plug-in name>-no-<option name>}. For instance, the
opposite of option \optionuse{-}{wp-print} is \optionuse{-}{wp-no-print}.
Use the options \texttt{-kernel-help} and
\texttt{-<plug-in name>-help} to get the opposite option name of each
parameterless option.

\paragraph{String Options}

If the option's argument is a
string (that is, neither an integer nor a float, \etc), the following
format is also possible:
\begin{commands}
\texttt{-option\_name=value}
\end{commands}

\begin{important}
This format \textbf{must be used} when \texttt{value} starts with a minus sign.
\end{important}

\paragraph{Set Options}

Some options (e.g. option \optionuse{-}{cpp-extra-args}) accept a set of
comma-separated values as argument. Each value may be prefix by \texttt{+}
(resp. \texttt{-}) to indicate that this value must be added to (resp. deleted
from) the set. When neither is specified, \texttt{+} is added by default.

As for string options, the extended format is also possible:
\begin{commands}
\texttt{-option\_name=values}
\end{commands}

\begin{important}
This format \textbf{must be used} if your argument contains a minus sign.
\end{important}

For instance, you can ask the \C preprocessor to search for header files in
directory \texttt{src} by setting:
\begin{commands}
\texttt{-cpp-extra-args="-I src"}
\end{commands}

Categories are specific values which describe a subset of values. Their names
begin with an \texttt{@}. Available categories are option-dependent, but most
set options accept the category \texttt{@all} which defines the set of all
acceptable values.

For instance, you can ask the \Value plug-in to use the ACSL specification of
each function but \texttt{main} instead of their definitions by setting:
\begin{commands}\optionidx{-}{eva-use-spec}
\texttt{-eva-use-spec="@all, -main"}
\end{commands}

\begin{important}
If the first character of a set value is either \texttt{+}, \texttt{-},
\texttt{@} or $\backslash$, it \textbf{must be escaped} with a $\backslash$.
\end{important}

\paragraph{Map Options}

Map options are set options whose values are of the form \texttt{key:value}. For
instance, you can override the default \Value's $slevel$~\cite{value} for
functions \texttt{f} and \texttt{g} by setting:
\begin{commands}\optionidx{-}{slevel-function}
\texttt{-slevel-function="f:16, g:42"}
\end{commands}

\subsection{Autocompletion for Options}

The \texttt{autocomplete\_frama-c} file in the directory of shared resources
(see \texttt{-print-share-path} option above) contains a
bash autocompletion script for \FramaC's options. In order to take advantage
of it, several possibilities exist.
\begin{itemize}
\item In order to enable system-wide completion, the file can be copied
  into the directory \texttt{/etc/bash\_completion.d/} where bash search for
  completion scripts by default
\item If you only want to add completion for yourself, you can append the
content of the file to \texttt{\~{}/.bash\_completion}
\item You can \texttt{source} the file, e.g. from your \texttt{.bashrc} with
the following command:
\end{itemize}
\begin{verbatim}
source $(frama-c-config -print-share-path)/autocomplete_frama-c || true
\end{verbatim}


There is also an autocompletion script for zsh, \texttt{\_frama-c}, also
in the shared resources directory. Look inside for installation instructions.

The kernel option \optiondef{-}{autocomplete} provides a text output listing
all \FramaC options, so that autocompletion scripts may use it to provide
completion. The \FramaC team welcomes improved and additional autocompletion
scripts.

\subsection{Splitting a \FramaC Execution into Several Steps}\label{sec:then}

By default, \FramaC parses its command line in an \emph{unspecified} order and
runs its actions according to the read options. To enforce an order of
execution, you have to use the option \optiondef{-}{then}: \FramaC parses its
command line until the option \texttt{-then} and runs its actions accordingly,
\emph{then} it parses its command line from this option to the end (or to the
next occurrence of \texttt{-then}) and runs its actions according to the read
options. Note that this second run starts with the results of the first one.

Consider for instance the following command.
\begin{frama-c-commands}
\$ frama-c -eva -ulevel 4 file.c -then -ulevel 5
\end{frama-c-commands}
It first runs the value analysis plug-in (option
\optionuse{-}{eva},~\cite{value}) with an unrolling level of 4 (option
\optionuse{-}{ulevel}, Section~\ref{sec:normalize}). Then it re-runs the value
analysis plug-in (option \optionuse{-}{eva} is still set) with an unrolling
level of 5.

It is also possible to specify a project (see Section~\ref{sec:project}) on
which the actions are applied thanks to the option \optiondef{-}{then-on}.
Consider for instance the following command.
\begin{frama-c-commands}
\$ frama-c -semantic-const-fold main file.c -then-on propagated -eva
\end{frama-c-commands}
It first propagates constants in function \texttt{main} of \texttt{file.c}
(option \optionuse{-}{semantic-const-fold}) which generates a new project
called \texttt{propagated}. Then it runs the value analysis plug-in on this new
project. Finally it restores the initial default project, except if the option
\optiondef{-}{set-project-as-default} is used as follows:
\begin{frama-c-commands}
\$ frama-c -semantic-const-fold main file.c \
          -then-on propagated -eva -set-project-as-default
\end{frama-c-commands}

Another possibility is the option \optiondef{-}{then-last} which applies the
next actions on the last project created by a program transformer. For
instance, the following command is equivalent to the previous one.
\begin{frama-c-commands}
\$ frama-c -semantic-const-fold main file.c -then-last -eva
\end{frama-c-commands}

The last option is \optiondef{-}{then-replace} which behaves like
\optionuse{-}{then-last} but also definitively destroys the previous current
project. It might be useful to prevent a prohibitive memory consumption. For
instance, the following command is equivalent to the previous one, but also
destroys the initial default project.
\begin{frama-c-commands}
\$ frama-c -semantic-const-fold main file.c -then-replace -eva
\end{frama-c-commands}

\subsection{Verbosity and Debugging Levels}

The \FramaC kernel and plug-ins usually output messages either in the GUI or
in the console. Their levels of verbosity may be set by using the option
\texttt{\optiondef{-}{verbose} <level>}. By default, this level is
1. Setting it to 0 limits the output to warnings and error messages, while setting it
to a number greater than 1 displays additional informative message (progress
of the analyses, \etc).

In the same fashion, debugging messages may be printed by using the option
\texttt{\optiondef{-}{debug} <level>}. By default, this level is 0: no
debugging message is printed. By contrast with standard messages, debugging
messages may refer to the internals of the analyzer, and may not be
understandable by non-developers.

The option \optiondef{-}{quiet} is a shortcut for \texttt{-verbose 0
  -debug 0}.

In the same way that \texttt{-verbose} (resp. \texttt{-debug}) sets the level
of verbosity (resp. debugging), the options
\optiondef{-}{kernel-verbose}
(resp. \optiondef{-}{kernel-debug}) and
\texttt{-<plug-in shortname>-verbose} (resp. \texttt{-<plug-in shortname>-debug}) set the
level of verbosity (resp. debugging) of the kernel and particular
plug-ins. When both the global level of verbosity (resp. debugging) and a
specific one are modified, the specific one applies. For instance,
\texttt{-verbose 0 -slicing-verbose 1} runs \FramaC quietly except for the
slicing plug-in.

It is also possible to choose which categories of message should be displayed
for a given plugin. See section~\ref{sec:feedback-options} for more information.

\subsection{Copying Output to Files}

Messages emitted by the logging mechanism (either by the kernel or by plug-ins)
can be copied to files using the \texttt{-<plug-in shortname>-log} option
(or \texttt{-kernel-log}), according to the following syntax:
\optionidxdef{-}{kernel-log}
\optionidxdef{-}{<plug-in>-log} \texttt{kinds1:file1,kinds2:file2,...}

Its argument is a map from {\em kind} specifiers to output files, where
each key is a set of flags defining the kind(s) of
message to be copied, that is, a sequence of one or several of the following
characters:
\begin{description}
\item\texttt{a}: all (equivalent to \texttt{defrw} or \texttt{dfiruw})
\item\texttt{d}: debug
\item\texttt{e}: user or internal error (equivalent to \texttt{iu})
\item\texttt{f}: feedback
\item\texttt{i}: internal error
\item\texttt{r}: result
\item\texttt{u}: user error
\item\texttt{w}: warning
\end{description}
If \texttt{kinds} is empty (e.g. \texttt{:file1}), it defaults to \texttt{erw},
that is, copy all but debug and feedback messages.

\texttt{file1}, \texttt{file2}, etc. are the names of the files where each
set of messages will be copied to.
Each file will be overwritten if existing, or created otherwise.
Note that multiple entries can be directed to a single file
(e.g.~\texttt{-kernel-log w:warn.txt -wp-log w:warn.txt}).
%Also note that the use of symbolic links can cause some issues:
%if \texttt{a.txt} is a symbolic link to \texttt{b.txt}, then you should
%never use e.g. \texttt{-plugin1-log :a.txt -plugin2-log :b.txt}.
%Otherwise, the behavior of this option is unspecified.

Here is an example of a sophisticated use case for this option:

\texttt{frama-c -kernel-log ew:warn.log -wp-log ew:warn.log \\
  \phantom{\qquad} -metrics-log :../metrics.log} [...] \texttt{file.c}

The command above will run some unspecified analyses ([...]), copying
the error and warning messages produced by both the kernel and the
\texttt{wp} plug-in into file \texttt{warn.log}, and copying all non-debug,
non-feedback output from the \texttt{metrics} plug-in into file
\texttt{../metrics.log} (note that there is a separator (\texttt{:}) before the
file path).

This option does not suppress the standard \FramaC output, but
only copies it to a file. Also, only messages which are effectively printed
(according to the defined verbosity and debugging levels) will be copied.

\subsection{Terminal Capabilities}

Some plug-ins can take advantage of terminal capabilities to enrich
output.  These features are automatically turned off when the \FramaC
standard output channel is not a terminal, which typically occurs when you
redirect it into a file or through a pipe.

You can control use of terminal capabilities with option
\optiondef{-}{tty}, which is set by default and can be deactivated
with \texttt{-no-tty}.

\subsection{Getting time}

The option \texttt{\optiondef{-}{time} <file>} appends user time and date to
the given log \texttt{<file>} at exit.

\subsection{Inputs and Outputs of Source Code}\label{sec:io}

The following options deal with the output of analyzed source code:
\begin{description}
\item \optiondef{-}{print} causes \FramaC's representation for the analyzed
  source files to be printed as a single C program (see
  Section~\ref{sec:normalize}). Note that files from the \FramaC standard
  library are kept by default under the form of \texttt{\#include} directives,
  to avoid polluting the output. To expand them, use \optiondef{-}{print-libc}.
\item \texttt{\optiondef{-}{ocode} <file name>} redirects all output code of the
  current project to the designated file.
\item \optiondef{-}{keep-comments} keeps C comments in-lined in the code.
\item \optiondef{-}{unicode} uses unicode characters in order to display some
  \acsl symbols. This option is set by default, so one usually uses the opposite
  option \texttt{-no-unicode}.
\end{description}
A series of dedicated options deal with the display of floating-point and
integer numbers:
\begin{description}
\item \optiondef{-}{float-hex} displays floating-point
  numbers as hexadecimal
\item \optiondef{-}{float-normal} displays floating-point numbers
  with an internal routine
\item \optiondef{-}{float-relative} displays intervals of
  floating-point numbers as \texttt{[lower bound ++ width]}
\item \texttt{\optiondef{-}{big-ints-hex} <max>} prints all integers greater
  than \texttt{max} (in absolute value) using hexadecimal notation
\end{description}

\section{Environment Variables}\label{sec:env-variables}

Different environment variables may be set to customize \FramaC.

\subsection{Variable \texttt{FRAMAC\_SESSION}}\label{sec:var-session}

\FramaC may have to generate files depending on the project under analysis
during a session in order to reuse them later in other sessions.

By default, these files are generated or searched in the subdirectory
\texttt{.frama-c} of the current directory. You can also set the environment
variable \textttdef{FRAMAC\_SESSION} or the option \optiondef{-}{session} to
change this path.

Each \FramaC plug-in may have its own session directory (default is
\texttt{.frama-c/<plug-in shortname>}). It is also possible to change
a plug-in's session directory by using the option \texttt{-<plug-in
  shortname>-session}.

\subsection{Variable \texttt{FRAMAC\_CONFIG}}\label{sec:var-config}

\FramaC may have to generate configuration files during a session in order to
reuse them later in other sessions.

By default, these files are generated or searched in a subdirectory
\texttt{frama-c} (or \texttt{.frama-c}) of the system's default configuration
directory ({\it e.g.}
\texttt{\%USERPROFILE\%} on Windows or \texttt{\$HOME/.config}
on Linux). You can also set the environment variable \textttdef{FRAMAC\_CONFIG}
or the option \optiondef{-}{config} to change this path.

Each \FramaC plug-in may have its own configuration directory if required (on
Linux, default is \texttt{\$HOME/.config/frama-c/<plug-in shortname>}). It is
also possible to change a plug-in's config directory by using the option
\texttt{-<plug-in shortname>-config}.

\section{Exit Status}

When exiting, \FramaC has one of the following status:
\begin{description}
\item[0] \FramaC exits normally without any error;
\item[1] \FramaC exits because of invalid user input;
\item[2] \FramaC exits because the user kills it (usually \via \texttt{Ctrl-C});
\item[3] \FramaC exits because the user tries to use an unimplemented feature.
  Please report a ``feature request'' on the Bug Tracking System (see
  Chapter~\ref{user-errors});
\item[4,5,6] \FramaC exits on an internal error. Please report a ``bug report'' on
  the Bug Tracking System (see Chapter~\ref{user-errors});
%\item[5] \FramaC raises an error while exiting normally.
%\item[6] \FramaC raises an error while exiting abnormally.
\item[125] \FramaC exits abnormally on an unknown error. Please report a ``bug
  report'' on the Bug Tracking System (see Chapter~\ref{user-errors}).
\end{description}

%% Local Variables:
%% compile-command: "make"
%% ispell-local-dictionary: "english"
%% TeX-master: "userman.tex"
%% End:
