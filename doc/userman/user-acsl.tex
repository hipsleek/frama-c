\chapter{ACSL extensions} % here we do not use the macro (avoids a warning)
\label{cha:acsl-extensions}
\section{Extension syntaxes}
\label{acsl:syntax}

When a plug-in registers an extension, it can be used in \acsl annotations
with 2 different syntaxes. As an example, with an extension \lstinline|bar|
registered by the plug-in \lstinline|foo|, we can use the short syntax
\lstinline|bar _| or the complete syntax \lstinline|\foo::bar _|.

The complete syntax is useful to print better warning/error messages, and to
better understand which plug-in introduced the extension. Additionnaly, all
extensions coming from an unloaded plug-in can be ignored this way. For
example, if \Eva is not loaded, \lstinline|\eva::unroll _| annotations will be
ignored with a warning, whereas \lstinline|unroll _| cannot be identified as
being supported by Eva, which means that it can only be treated as a user
error. 

\section{Handling indirect calls with \texttt{calls}}
\label{acsl:calls}

In order to help plug-ins support indirect calls (i.e. calls through
a function pointer), an \acsl extension is provided. It is introduced
by keyword \lstinline|calls| and can be placed before
a statement with an indirect call to give the list of
functions that may be the target of the call. As an example,
\begin{ccode}
/*@ calls f1, f2, ... , fn */
*f(args);
\end{ccode}
indicates that the pointer \lstinline|f| can point to any one of
\lstinline|f1|, \lstinline|f2|, ..., \lstinline|fn|.

It is in particular used by the WP plug-in (see \cite{wp} for more information).

%%% Local Variables:
%%% mode: latex
%%% TeX-master: "userman"
%%% End:
